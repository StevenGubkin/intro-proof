\documentclass{article}


\usepackage{geometry}                % See geometry.pdf to learn the layout options. There are lots.
\geometry{letterpaper, margin=0.7in}                   % ... or a4paper or a5paper or ... 
%\geometry{landscape}                % Activate for for rotated page geometry
\usepackage[parfill]{parskip}    % Activate to begin paragraphs with an empty line rather than an indent
\usepackage{graphicx}
\usepackage{amssymb}
\usepackage{epstopdf}
\usepackage{pdfpages}
\usepackage{colortbl}
\usepackage{amsmath}
\usepackage[linguistics]{forest}
\usepackage{fitch}
\usepackage{amsthm}
\usepackage{booktabs}
\newcolumntype{?}{!{\vrule width 1pt}}
\newtheorem*{Theorem}{Theorem}


\usepackage{hyperref}

\usepackage{tikz}
\newcommand*\circled[1]{\tikz[baseline=(char.base)]{
  \node[shape=circle,draw,inner sep=2pt] (char) {#1};}}

\DeclareGraphicsRule{.tif}{png}{.png}{`convert #1 `dirname #1`/`basename #1 .tif`.png}

\renewcommand{\arraystretch}{1.75}

\usepackage{lipsum}   % for filler text
\usepackage{setspace} % for \onehalfspacing and \singlespacing macros
%\onehalfspacing 

\title{Fitch Style Proof Outlines}
\author{Steven Gubkin}
\date{}
\usepackage{etoolbox}


\newcommand{\equivalent}{\Longleftrightarrow}
\newcommand{\Z}{\mathbb{Z}}


\begin{document}

\begin{center}
	{\huge L3 summary table}
	\end{center}

\begin{table}[h]
	\centering
	\begin{tabular}{c?p{5 cm}?p{4 cm}}
		Top Level Node & How to use it if it is true & How to prove that it is true	\\ \specialrule{.15em}{.05em}{.05em} 
		$p \wedge q$ & You also know that $p$ is true and $q$ is true.  & Argue $p$ is true.  Then argue $q$ is true. \\ \hline
		$p \implies q$ & If you also know $p$ is true, then you know $q$ is true. & \begin{fitch*}
				\textrm{Assume $p$ is true.} \\
				\fa \textrm{Argue that $q$ is true.}
			\end{fitch*}\\ \hline
		$p \equivalent q$ & If you know the truth value of one, then the truth value of the other is identical. &  
		\begin{fitch*}
				\textrm{Assume $p$ is true.} \\
				\fa \textrm{Argue that $q$ is true.}\\
				\textrm{Assume $q$ is true.} \\
				\fa \textrm{Argue that $p$ is true.}
		\end{fitch*}
		\\ \hline
		$\neg p$ & $p$ is false. &  
		\begin{fitch*}
			\textrm{Assume $p$}\\
			\fa \textrm{Argue $F$}
			\end{fitch*}
		\\ \hline
		$p \vee q$ & If you are trying to argue $r$ is true, then you must split your argument into cases.
		\begin{fitch*}
			\textrm{Case 1: Assume $p$ is true.}\\
			\fa \textrm{Argue $r$ is true.}\\
			\textrm{Case 2:  Assume $q$ is true.}\\
			\fa \textrm{Argue $r$ is true.}
			\end{fitch*}
		 &  Argue $p$ is true.
		 
		 	\medskip
		 
		 	OR
		 
		 	\medskip
		 
		     Argue $q$ is true.
		 
		     \medskip
		 
		     OR
		
		 \begin{fitch*}
		 	\textrm{Assume $\neg p$ is true.}\\
		 	\fa \textrm{Argue $q$ is true. }
		 	\end{fitch*}
	 	
	 	OR 
	 	
	 			 \begin{fitch*}
	 		\textrm{Assume $\neg q$ is true.}\\
	 		\fa \textrm{Argue $p$ is true. }
	 	\end{fitch*}

		 \\ \hline
		$\forall x, A(x)$ & If $t$ is any element of the universe of discourse, you know that $A(t)$ is true. &  Let $x_1$ be an arbitrary element of the universe of discourse, and argue that $A(x_1)$ is true. \\ \hline
		$\exists x, B(x)$ & You know that there is at least one (and potentially only one) element of your universe of discourse which makes $B$ true.  Pick one and call it $x_1$.  You may freely use that $B(x_1)$ is true in your argument. &  You need to come up with a candidate $x_1$ for which you think $B(x_1)$ is true.  Often, this will be a formula in terms of previously declared variables.  Then argue that $B(x_1)$ really is true.
	\end{tabular}
\end{table}


	
\end{document}




