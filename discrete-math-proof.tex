\documentclass[multixcb]{amstext-l}

\usepackage{amssymb}
\usepackage{imakeidx}
\usepackage{graphicx}
\usepackage{natbib}

\usepackage[linguistics]{forest}
\usepackage{fitch}
\usepackage{listings}


\newtheorem{theorem}{Theorem}[chapter]
\newtheorem{conjecture}{Conjecture}
\newtheorem{lemma}[theorem]{Lemma}

\theoremstyle{definition}
\newtheorem{definition}[theorem]{Definition}
\newtheorem{example}[theorem]{Example}
\newtheorem{xca}[theorem]{Exercise}
\newtheorem{solutions}[theorem]{Solutions}

\theoremstyle{remark}
\newtheorem{remark}[theorem]{Remark}

\numberwithin{section}{chapter}
\numberwithin{equation}{chapter}

\newcommand{\equivalent}{\Longleftrightarrow}
\newcommand{\Z}{\mathbb{Z}}
\newcommand{\T}{\textbf{\textrm{T}}}
\newcommand{\F}{\textbf{\textrm{F}}}
\newcolumntype{?}{!{\vrule width 1pt}}

\makeindex

\begin{document}

\frontmatter

\title{Discrete Mathematics:  An Introduction to Mathematical Reasoning}

\author{Steven Gubkin}
\address{5410 Lockwood Road, Madison, Ohio, 44057}
\email{s.gubkin@csuohio.edu}

%\subjclass[2020]{Primary }

%\keywords{}

\maketitle

%    Dedication.  If the dedication is longer than a line or two,
%    remove the centering instructions and the line break.
%\cleardoublepage
%\thispagestyle{empty}
%    If this book uses the documentclass stml-l or mmono-s, change
%    13.5pc to 10.5pc.
%\vspace*{13.5pc}
%\begin{center}
%  Dedication text (use \\[2pt] for line break if necessary)
%\end{center}
%\cleardoublepage

%    Change page number to 7 if a dedication is present.
\setcounter{page}{5}

\tableofcontents

%    Include unnumbered chapters (preface, acknowledgments, etc.) here.
%     \include{}

\mainmatter
%    Include main chapters here.
%\chapter{Propositions, Predicates, and Quantifiers}

\section{Introduction}

The main goal of this text is to help you create and critique logical arguments in mathematics.  We will need to develop extremely precise vocabulary since ordinary language is not sufficiently precise to be suitable for our purposes.  So this text is, in addition to being a mathematics text, also a kind of foreign language text.  We will learn to distinguish special kinds of sentences called \textbf{propositions} ( assertions which are either true or false ) from other sentences.  The sentence ``6 is larger than 8'' is a proposition (a false one), while the sentence ``Please sum 6 with 8'' is not a proposition.   We will learn how to handle sentences which contain variables (these are called \textbf{predicates}).  A predicate might be true or false depending on the value of its variables. ``$2x$  is less than $y$'' is a predicate which is true when $(x,y) = (1,20)$, but is false when $(x,y) = (4,7)$.  When we make a claim about the quantity of variables which make a predicate true we are \textbf{quantifying} the predicate.  In the sentence ``there is at least one value of $x$ which makes $x^2  = 9$'', the predicate ``$x^2 =9$'' has been quantified with the statement that ``At least one value of $x$ makes the predicate true''.  

Towards the end of the chapter we will also learn the rules for \textbf{using} and \textbf{proving} quantified statements.  Eventually we will want to be able to analyze the logical structure of any statement we want to prove and create a \textbf{natural deduction outline} for our argument.  This is basically a sketch of what is required to argue a statement based exclusively on its logical form (not its content).  We will learn how to start a natural deduction outline for quantified sentences.

In addition to being essential for making mathematical arguments, the material in this chapter will also be essential to anyone who needs to program computers.  Computers follow instructions literally, and we often need these instructions to vary depending on whether some predicate is true or false given the value of a variable.  Consider the simple problem of instructing a computer to sum all of the numbers between $1$ and $100$ which are divisible by $6$.

In psuedocode, we might write

\begin{lstlisting}[language=Python]
	set SUM = 0
	for k in {1, 2, 3, ..., 100}:
		If (k%6 == 0):
			SUM := SUM + k 
	return SUM
\end{lstlisting}

This code depends on our ability to understand that the predicate $k\%6 == 0$ is true when $k$ is divisible by $6$, and false otherwise.  We then need to understand that we will only add $k$ to the previous sum when this predicate is true.

For this reason, having a very detailed understanding of propositions, predicates, and quantifiers is also essential for computer programmers.

\section{A short note on Sets}

In this chapter we will not need to know very much about sets, but we will need a few   essentials.  We will conduct a more in depth study of sets, including operations on sets, in a later chapter.

\begin{definition}
		A \index{Set}\textbf{set} is a collection of objects which we  call \index{Element of a Set} \textbf{elements}.  If $x$ is an element of $S$ we write $x \in S$ and say ``$x$ is an element of the set $S$''.  We consider two sets $A$ and $B$ to be equal if every element of $A$ is also in $B$, and every element of $B$ is also in $A$. 
	\end{definition}

To specify a set, we write its elements separated by commas between curly braces.  For instance, to specify the set of siblings in my family I would write:

\[
\textrm{Siblings} = \{\textrm{Steve, Kayla, Peter, Orin}\}
\]

If I write $\textrm{Kayla} \in \textrm{Siblings}$ I am indicating that Kayla is in the set Siblings.  If I write $\textrm{Mike} \notin \textrm{Siblings}$ this indicates that Mike is not in the set Siblings.

When we defined set equality, we said ``We consider two sets $A$ and $B$ to be equal if every element of $A$ is also in $B$, and every element of $B$ is also in $A$. ''

Let us explore the consequences of this definition.

Should we consider the set $\{4,3,2\}$ to be the same as the set $\{3,2,4\}$?  Ask yourself:  is every element of one set also an element of the other?  The answer is yes:  we should consider these sets to be equivalent to each other.  We sometimes shorthand this observation by saying that ``sets do not care about the order in which their elements are listed''.  In other words, the ordering of the elements in the written expression ``$\{4,3,2\}$'' is a feature of the written representation, not a feature of the set itself.  This is similar to how  ``Steven Gubkin'' and ``\textit{Steve Gubkin}'' refer to the same person despite typographical differences.

Should we consider the set $\{4,5,6,5\}$ to be the same as the set $\{6,5,4\}$?  Again ask yourself:  is every element of one set also present in the other?  Again the answer is yes, so by definition these two sets should be considered equal to each other.  The fact that $5$ is listed twice in our written representation of the set does not change this fact.  We sometimes shorthand this by saying ``Sets do not have repeated elements''.

There are a few special sets which you should know the names and notation for:

\begin{itemize}
		\item The set of \index{Natural Numbers}\textbf{natural numbers}\footnote{Some mathematicians prefer to exclude $0$ from their definition of the natural numbers.  This is a matter of taste, and people have very heartfelt reasons for their choice.  This text chooses to include $0$ in the set of natural numbers.}  is the set
		
		\[
		\mathbb{N} = \{0,1,2,3,4, \dots \}
		\]
		
		\item The set of \index{Integers}\textbf{integers} is the set
		
		\[
		\mathbb{Z} = \{\dots, -3,-2,-1,0,1,2,3,\dots\}
		\]
		
		\item The set of \index{Rational Numbers}\textbf{rational numbers} consists of all numbers which can be expressed in the form $\frac{a}{b}$ where $a$ and $b$ are integers with $b \neq 0$.  Some representative elements are shown in curly braces below.
		
		\[
		\mathbb{Q} = \{ \dots,  -\frac{111}{25},-4,0,\frac{2}{7}, \frac{1}{2}, 2, 2.37, 2.\bar{5}, 3, \frac{100}{9}, \dots\}
		\] 
		
		\item The set of \index{Real Numbers}\textbf{real numbers} is hard to define.  You will not meet a modern rigorous definition of a real number unless you study a subject called real analysis.  Intuitively, the real numbers consist of all rational numbers and also limits of sequences of rational numbers.  For instance, $3.141$ is a rational number because $3.141 = \frac{3141}{1000}$.  $\pi$ is a real number because it can be thought of as a limit of the sequence of rational numbers 
		
		\[3, 3.1, 3.14, 3.141, 3.1415, \cdots\]
		
		You can think of a real number as a number with a potentially infinite decimal expansion which may or may not repeat.  It includes positive numbers, negative numbers, and zero.
		
		\[
		\mathbb{R} = \{ \dots, -\sqrt{2}, -1, -\frac{1}{3}, 0, 5, 5.\overline{61}, \pi, \sqrt[3]{29}, \dots \}
		\]
		
		\item The set of \index{Complex Numbers}\textbf{complex numbers} consists of numbers of the form $a+bi$ where $a,b \in \mathbb{R}$ and $i$ is one of the two square roots of negative one.  Every real number is also a complex number.  For example, $\pi = \pi + 0i$.
		
		\[
		\mathbb{C} = \{\cdots, -3+2i, \pi+\sqrt{2}i, ,0, 5, \frac{1}{2} - \frac{3}{4} i, \dots \}
		\]
	\end{itemize}

\begin{xca}
		Which of the following are true and which are false?
			
		\begin{enumerate}
			\item $\{2,3\} = \{3, 2,3\}$
			\item $\{2,3\} = \{4,3,2\}$
			\item $\{7,3,2\} = \{3,7,2,2\}$
			\item $4 \in \{1,2,3,4,5\}$
			\item $5 \in \{0,1,2,3\}$
			\item $7 \notin \{0,1,2, \dots, 20\}$
			\item $8 \in \{0,1,2, \dots,  9\}$
			\item $4.2 \in \mathbb{N}$
			\item $4.2 \in \mathbb{Z}$
			\item $4.2 \in \mathbb{Q}$
			\item $4.2 \in \mathbb{R}$
			\item $4.2 \in \mathbb{C}$
			\end{enumerate}
		
\end{xca}

\begin{solutions}
	Which of the following are true and which are false?
	
	\begin{enumerate}
		\item $\{2,3\} = \{3, 2,3\}$ - True.  $2$ and $3$ are the elements of the first set, and they are also elements of the second set.  Conversely, each element of the second set $3$,  $2$, and $3$ is also an element of the first set. 
		\item $\{2,3\} = \{4,3,2\}$  - False.  $4$ is an element of the second set, but not the first.
		\item $\{7,3,2\} = \{3,7,2,2\}$ - True.
		\item $4 \in \{1,2,3,4,5\}$ - True.
		\item $5 \in \{0,1,2,3\}$ - False
		\item $7 \notin \{0,1,2, \dots, 20\}$ - False.  While $7$ is not explicitly listed in the second set, we are supposed to infer from the ellipses that every natural number between $2$ and $20$ is also included in the set.  So $7$ is in there!
		\item $8 \in \{0,1,2, \dots,  9\}$ - True.
		\item $4.2 \in \mathbb{N}$ - False.  $4.2$ is not a natural number.
		\item $4.2 \in \mathbb{Z}$ - False.  $4.2$ is not an integer.
		\item $4.2 \in \mathbb{Q}$ - True.  $4.2 = \frac{42}{10} = \frac{21}{5}$, so it can be written as a ratio of two integers.  Thus $4.2$ is rational.
		\item $4.2 \in \mathbb{R}$ - True.  Every rational number is real.  To be pedantic about it, we could think of $4.2$ as the limit of the sequence of rational numbers $4.2, 4.2, 4.2, 4.2, \dots$.
		\item $4.2 \in \mathbb{C}$ - True.  $4.2 = 4.2+0i$, so it is also a complex number.
	\end{enumerate}
\end{solutions}

\begin{xca}
		For each of the following sets, list $3$ things which are elements of the set and $3$ things which are not an element of the set.
		
		\begin{enumerate}
				\item $S = \{\textrm{a, c, e, g, i, k, m, o, q, s, u, w, y}\}$
				\item The set $\mathcal{P}$ of humans who have ever lived.
				\item  $\{-11,-10, -9,  \dots, 10, 11, 12\}$
				\item $\mathbb{N}$
				\item $\mathbb{Z}$
				\item $\mathbb{Q}$
				\item $\mathbb{R}$
				\item $\mathbb{C}$
			\end{enumerate}
	\end{xca}

\begin{solutions}
	For each of the following sets, list $3$ things which are elements of the set and $3$ things which are not an element of the set.
	
	\begin{enumerate}
		\item $S = \{\textrm{a, c, e, g, i, k, m, o, q, s, u, w, y}\}$ - a, g, and i are all elements of $S$, while b, d, 7, and my dog are not.   
		\item The set $\mathcal{P}$ of all humans. -  George Washington, Celine Dion, and Cher are all in the set $\mathcal{P}$.  $3+2i$, my dog, and the sun are not elements of $\mathcal{P}$.
		\item  $\{-11,-10, -9,  \dots, 10, 11, 12\}$ - This set includes $-4$, $0$, and $9$.  It does not include $13$, $-100$, or $1.9$.
		\item $\mathbb{N}$ - Includes $0$, $1$, and $99$.  Does not include $-5$, $\frac{1}{2}$, or $\pi$.
		\item $\mathbb{Z}$ - Includes $0$, $\frac{6}{2}$, and $-1032$.  Does not include $\frac{7}{2}$, $\sqrt{7}$, or $7-i$.
		\item $\mathbb{Q}$ - Includes $0$, $-100.1$, and $\frac{65}{3}$.  Does not include $\sqrt{2}$, $e$, or $0.12345678910111213...$.  However, proving that these numbers are not rational is actually fairly difficult.
		\item $\mathbb{R}$ - Includes $e$, $7$, $-\frac{1}{2}$.  Does not include $3+2i$, $\frac{0}{0}$, or $\infty$.
		\item $\mathbb{C}$ - Includes $e$, $2.2 - 1.7i$, $22i$.  Does not include $\frac{4}{0}$, $\infty$, or the ordered pair $(3,2)$.
	\end{enumerate}
\end{solutions}




\section{Propositions}

\begin{definition}
		A \index{proposition}\textbf{proposition} is a sentence which is either true or false.
\end{definition}

\begin{xca}
		Which of the following sentences are propositions?  If the sentence is a proposition decide its truth value (``true" or ``false'').  If the sentence is not a proposition, explain why not.
		\begin{enumerate}
				\item ``$7$ is a prime number.''
				\item ``$6$ is a prime number."
				\item ``What is an axolotl?"
				\item ``An axolotl is a small amphibian."
				\item ``Every integer is a prime number."
				\item ``Some integers are prime numbers."
				\item  $4 \in \mathbb{Z}$
				\item ``Fetch me some water."
				\item ``Here is some water."
				\item ``I got this water for you."
				\item ``$2+2 = 4$."
				\item ``$6 \cdot 7$."
				\item ``One billion is the largest number."
				\item ``George Washington (the U.S. president) was alive on September 5th, 1923."
				\item ``The sentence `$2+2 = 5$ is a proposition.'''
			\end{enumerate}
	\end{xca}

\begin{solutions}
		\begin{enumerate}
	\item ``$7$ is a prime number.'' -  This is true proposition.
	\item ``$6$ is a prime number." -  This is a false proposition.
	\item ``What is an axolotl?" - This is not a proposition, since it is not true or false.  It is a question.
	\item ``An axolotl is a small amphibian." -  This is a true proposition.
	\item ``Every integer is a prime number." -  This is a false proposition.
	\item ``Some integers are prime numbers." - This is a true proposition.
	\item  ``$4 \in \mathbb{Z}$'' - This is a true proposition.  It makes the claim that $4$ is an integer.
	\item ``Fetch me some water." -  This is not a proposition, since it is not true or false.  It is a command.
	\item ``Here is some water." -  This is not a proposition.  It is an offer, which is not true or false.
	\item ``I got this water for you." -  This is a proposition.  It could be true or false depending on my real intentions.
	\item ``$2+2 = 4$." -  This is a true proposition.  It is a sentence which makes the claim that two plus two is equal to four, which is true.
	\item ``$6 \cdot 7$." -  This is not a proposition.  It is the number $42$, which is just a number, not a claim which is true or false.
	\item ``One billion is the largest number." -  This is a false proposition.
	\item ``George Washington (the U.S. president) was alive on September 5th, 1923." -  This is a false proposition.
	\item ``The sentence `$2+2 = 5$' is a proposition.'' - This is a true proposition.  It makes a claim that the sentence `$2+2 = 5$' is a proposition.  Since `$2+2 = 5$' is a proposition (a false one), the sentence is true.
\end{enumerate}
	\end{solutions}

Note that we use quotes around propositions:  a proposition is a sentence, and we want to refer to the entire sentence by quoting it. 

In the last example, we had a proposition which referred to another proposition.  This kind of ``meta-level'' proposition will be extremely important to us when we get to quantification. 

In the future, we will need to work with many propositions at the same time.  It will sometimes be convenient to use upper case letters to stand for entire propositions.  So we might write

		\[P  =  \textrm{``The Sun revolves about the Earth.''} \]

In this particular case, we might also write $P = \F$ to indicate that $P$ is a false statement.  Otherwise, we would have written $P = \T$.  This is a slight abuse of the equals sign.  The proposition $P$ is not literally just the truth value $\F$.  We are using the equals sign to indicate that $P$ is being assigned the truth value $\F$.  This is a subtle distinction, but one worth thinking about.

We will often use $P$, $Q$, and $R$ as letters representing propositions.

Note that equations and inequalities are also examples of propositions. We can write

\begin{enumerate}
\item \(P = \textrm{``\(2 \cdot 3 = 6\)''}\).  This is a true proposition. The proposition is not the number $6$, but the claim that ``$2$ times $3$ is equal to $6$''.  Note that we are using the equals sign in two subtly different ways here:  the first equals sign is declaring that $P$ \textbf{is} the proposition which follows.  The second equals sign is asserting that $2 \cdot 3$ \textbf{is} the number 6.
\item \(Q = \textrm{``\(2 <  1\)''}\).  This is a false proposition. The proposition is the claim that ``$2$ is less than $1$".
\item  \(R = 2 \cdot 3\) is not a proposition at all.  It is just a number.  It doesn't make sense to claim that $2\cdot 3$ is true or false.
\end{enumerate}

\section{Predicates}

Consider

\begin{center}
``$x$ is a prime number''
\end{center}

This is not a proposition because we cannot tell if it is true or false until we know what $x$ is.  If $x$ is $5$, then this is a true sentence.  If $x$ is $4$, then the sentence is false.  The letter ``$x$'' is being used as a \index{variable}\textbf{variable} here.  

\begin{definition} 
A \index{predicate}\textbf{predicate} is a statement which has variables.  
\end{definition}

Just as we use upper case letters to stand for propositions, we also use upper case letters to stand for predicates.   We might write

\[
P(x,y) =  \textrm{``$x+ y$ is greater than $y$''} 
\]

Then $P(1,3)$ is true (since $1+3$ is greater than $3$) while $P(-1,3)$ is false (since $-1+3$ is less than $3$).

We could also write this predicate entirely with symbols as 

\[
P(x,y) =  \textrm{``$x+ y > y$ ''} 
\]

The variables of a predicate will often be restricted to be of a certain \index{type}\textbf{type}.  For instance, in the proposition $P$ above, it makes sense for $x$ and $y$ to be real numbers, but it doesn't make sense for $x$ or $y$ to be complex numbers (since there is no sensible way to define ``less than'' or ``greater than'' for complex numbers).  Even worse would be trying to set $x = \textrm{``An apple''}$ and $y = \textrm{``An orange''}$.  There is no sensible way to add or compare these fruits.  Often the type of the variables in a predicate will be clear from context, but it is sometimes helpful to declare the type of the variables.  Here we might say

\[
P(x,y) =  \textrm{``$x+ y > y$ '', where $x,y \in \mathbb{R}$} 
\]

if we wanted to clarify.

It is also possible to have a predicate whose variables have different types.  Consider the predicate

\begin{align*}
M(f,x) &=  \textrm{``$f$ achieves its maximum value at input $x$'',}\\ 
	&  \textrm{ \hphantom{fdsfdsds} where $x\in \mathbb{R}$ and $f: \mathbb{R} \to \mathbb{R}$ is a function} \\
	& \textrm{ \hphantom{fdsfdsds} which takes real inputs and returns real outputs.}
\end{align*}

The first argument of $M$ is a function, while the second argument is a real number.


\begin{xca} In this exercise, allow the variables to stand for people who are currently living.  For each of the following statements, decide which are predicates.  For those which are predicates find some assignments of the variables which yield a true statement and some assignments of the variables which yield a false statement.  You may need to conduct an internet search to find examples of people satisfying the criteria.
		\begin{enumerate}
	\item ``$x$ is more than $110$ years old.'' - 
	\item $x$ and Bob.
	\item $y$ is able to sprint at $20$ miles per hour.
	\item Tell $z$ to go to the store.
	\item $x$ is taller than $y$.
	\item Where is $x$?
	\item $x$ is the mother of $y$.
\end{enumerate}

\end{xca}

\begin{solutions}


	\begin{enumerate}
	\item ``$x$ is more than $110$ years old.'' - This is a predicate.  It is true when $x$ is Kane Tanaka, at of this writing (May 17th, 2021).  It is false when $x$ is the author of this textbook.
	\item ``$x$ and Bob.'' - This is not a predicate.  If we substitute a person for $x$, we obtain a sentence which is neither true nor false.
	\item ``$y$ is able to sprint at $20$ miles per hour.'' - This is a predicate.  It is true when $y$ is Usain Bolt, and is false when $y$ is the author of this textbook.
	\item ``Tell $z$ to go to the store.'' - This is not a predicate.  When $z$ is a person, the sentence is a command, not a true or false statement. 
	\item ``$x$ is taller than $y$.'' - This is a predicate.  It is true when we let $x$ be Forest Whitaker (6'2"), and $y$ is Halle Berry (5'5").  It is false if we let $x$ be Halle Berry and $y$ be Forest Whitaker.
	\item ``Where is $x$?'' - This is not a predicate.
	\item ``$x$ is the mother of $y$.'' - This is true when $x$ is Judith Cohen (Nasa engineer and author) and $y$ is Jack Black (Comedian and musician).  It is false when we reverse their roles.
\end{enumerate}
		
	\end{solutions}

\begin{xca}	
In this exercise, allow the variables to stand for real numbers. For each of the following statements, decide which are predicates. For those which are  predicates, find some assignments of the variables which yield a true statement and some assignments of the variables which yield a false statement.
	
\begin{enumerate}
			\item ``$x \geq 110$''
			\item ``$2x+y$''
			\item ``$y^2  = 100$''
			\item ``$x + y = 20 + y$''
			\item ``$x+4 = $''
			\item ``$x$ is more than twice $y$.''
			\item ``$x$ is in between $y$ and $z$.''
		\end{enumerate}
	\end{xca}

\begin{solutions}
	
\begin{enumerate}
	\item ``$x \geq 110$'' - This is a predicate.  It is true when $x = 111$ and false when $x = 109$.
	\item ``$2x+y$'' - This is not a predicate.  If we substitute numbers for $x$ and $y$ (like $x  = 3$  and $y=1$) then we obtain a number (like $7$), not a proposition.  A number is not making a claim which is true or false. 
	\item ``$y^2  = 100$'' - This is a predicate.  It is true when $y = -10$.  It is false when $y = 3$.
	\item ``$x + y = 20 + y$'' - This is a predicate.  It is true when $x = 20$ and $y=4$.  It is false when $x = 19$ and $y=4$.
	\item ``$x+4 = $'' - This is not a predicate.  It is kind of a sentence fragment.  If we substitute a number, we get something like ``3 plus 4 is....''. 
	\item ``$x$ is more than twice $y$.'' - This is a predicate.  It is true when $x = 10$ and $y=1$.  It is false when $x = y = 7$. 
	\item ``$x$ is in between $y$ and $z$.'' - This is a predicate.  It is true when $x = 2$, $y=1$, and $z = 3$.  It is false when $x = 100$, $y = 1$, and $z = \pi$.
\end{enumerate}
		
	\end{solutions}

\begin{xca}
	
	 For each predicate, evaluate the truth value of the indicated expression.
		
		\begin{enumerate}
				\item $P(x) = \textrm{``\(x > 5\)''}$
					\begin{enumerate}
							\item $P(0)$
							\item $P(4)$
							\item $P(5)$
							\item $P(6)$
						\end{enumerate}
				\item $Q(x,y) = \textrm{``\(x\) is one less than \(y\)''}$
						\begin{enumerate}
								\item $Q(1,3)$
								\item $Q(3,4)$
								\item $Q(4,3)$
								\item $Q(-\frac{1}{4}, \frac{3}{4})$
							\end{enumerate} 
				\item $R(x,y,z) = \textrm{``\( x \leq y < z\)''}$
					\begin{enumerate}
							\item $R(1,2,3)$
							\item $R(1,1,1)$
							\item $R(1,1,2)$
							\item $R(-1,0,1)$
							\item $R(-3,-2,-1)$
							\item $R(0,0,-1)$
						\end{enumerate}
				\item $S(x,y) = \textrm{``\(x = 4\)''}$
					\begin{enumerate}
							\item $S(2,3)$
							\item $S(4,2)$
							\item $S(2,4)$
							\item $S(4,4)$
						\end{enumerate}
			\end{enumerate}
	\end{xca}

\begin{solutions}
		\begin{enumerate}
		\item $P(x) = \textrm{``\(x > 5\)''}$
		\begin{enumerate}
			\item $P(0) = \textrm{``\(0 > 5\)''} = \F $
			\item $P(4) = \textrm{``\(4 > 5\)''} = \F$
			\item $P(5) = \textrm{``\(5 > 5\)''} = \F$
			\item $P(6) = \textrm{``\(6 > 5\)''} = \T$
		\end{enumerate}
		\item $Q(x,y) = \textrm{``\(x\) is one less than \(y\)''}$
		\begin{enumerate}
			\item $Q(1,3) = \textrm{``\(1\) is one less than \(3\)''} = \F$
			\item $Q(3,4)= \textrm{``\(3\) is one less than \(4\)''} = \T$
			\item $Q(4,3)= \textrm{``\(4\) is one less than \(3\)''} = \F$
			\item $Q(-\frac{1}{4}, \frac{3}{4}) = \textrm{``\(-\frac{1}{4}\) is one less than \(\frac{3}{4}\)''} = \T$
		\end{enumerate} 
		\item $R(x,y,z) = \textrm{``\( x \leq y < z\)''}$
		\begin{enumerate}
			\item $R(1,2,3)= \textrm{``\( 1 \leq 2 < 3\)''} = \T$
			\item $R(1,1,1)= \textrm{``\( 1 \leq 1 < 1\)''} = \F$, since $1$ is not strictly less than $1$.
			\item $R(1,1,2)= \textrm{``\( 1 \leq 1 < 2\)''} = \T$.
			\item $R(-1,0,1)= \textrm{``\( -1 \leq 0 < z1\)''} = \T$
			\item $R(-3,-2,-1)= \textrm{``\( -3 \leq -2 < z-1\)''} = \T$
			\item $R(0,0,-1)= \textrm{``\( 0 \leq 0 < -1\)''} = \F$
		\end{enumerate}
		\item $S(x,y) = \textrm{``\(x = 4\)''}$
		\begin{enumerate}
			\item $S(2,3) = \textrm{``\(2 = 4\)''} = \F$
			\item $S(4,2) = \textrm{``\(4 = 4\)''} = \T$
			\item $S(2,4) = \textrm{``\(2 = 4\)''} = \F$
			\item $S(4,4) = \textrm{``\(4 = 4\)''} = \T$
		\end{enumerate}
	\end{enumerate}
		
	\end{solutions}


As the last example indicates, it is sometimes convenient to think of a predicate as having more variables than are actually mentioned.  This is similar to how we still think of $f(x) = 5$ as a function of $x$, even though the output does not depend on the value of $x$.

It is also sometimes convenient to think of a proposition as a predicate with no variables!

Consider the following four predicates:

\begin{enumerate}
	\item $P(x,y,z) = \textrm{``\(x+y^2 = z\)''}$
	\item $Q(x,y) = \textrm{``\(x+y^2 = 10\)''}$
	\item $R(t) = \textrm{``\(t+7^2 = 10\)''}$
	\item $S = \textrm{``\(3+7^2 = 10\)''}$ (this is a false proposition)
	\end{enumerate}

In each case, we have obtained a new, but related, predicate from another by \index{substitution}\textbf{substitution}.

It is valid to write all of the following equations:

\begin{itemize}
		\item $Q(x,y) = P(x,y,10)$
		\item $R(t) = Q(t,7)$
		\item $R(t) = P(t,7,10)$
		\item $S = R(3)$
		\item $S = Q(3,7)$
	\end{itemize}

\begin{xca}
 Let $P$ be the two variable predicate defined by $P(a,b) = \textrm{``\(a^2b < 10\)''}$.  We can obtain other predicates by substitution.  For example $Q(x) = P(x,2x)$ is the predicate $\textrm{``\(x^2(2x) < 10\)''}$ which is equivalent to $x^3 < 5$.  Find inequalities which are equivalent to each of the following predicates.  If the inequality is a proposition, decide its truth value.  If the inequality is a predicate, find at least one assignment of the variables which makes the predicate true.
		\begin{enumerate}
				\item $Q(a) = P(a,5)$.  
				\item $Q = P(2,1)$.  
				\item $Q = P(2,3)$.
				\item $Q(t) = P(2,t)$
				\item $Q(c) = P(c,c)$
				\item $Q(a) = P(a,a)$
				\item $Q(a,b) = P(b,a)$
		\end{enumerate}
	\end{xca}

\begin{solutions}
	Let $P$ be the two variable predicate defined by $P(a,b) = \textrm{``\(a^2b < 10\)''}$.  We can obtain other predicates by substitution.  For example, $Q(x) = P(x,2x)$ is the predicate $\textrm{``\(x^2(2x) < 10\)''}$ which is equivalent to $x^3 < 5$.  Find inequalities which are equivalent to each of the following predicates.  If the inequality is a proposition, decide its truth value.  If the inequality is a predicate, find at least one assignment of the variables which makes the predicate true.
	\begin{enumerate}
		\item $Q(a) = P(a,5) = \textrm{``\(a^2b < 10\)''} = \textrm{``\((a)^2\cdot 5 < 10\)''} = \textrm{``\(a^2 < 2\)''}  $.  
		
		This is a predicate.  It is true when $a = 1$ and false when $a=2$.
		
		\item $Q = P(2,1) = \textrm{``\(2^2\cdot 1 < 10\)''} $.  This is a true proposition.  $4$ is less than $10$.
		\item $Q = P(2,3)= \textrm{``\(2^2 \cdot 3 < 10\)''} $.  This is a false proposition.  $12$ is not less than $10$.
		\item $Q(t) = P(2,t) = \textrm{``\(2^2\cdot t < 10\)''} =  \textrm{``\(t < \frac{5}{2}\)''}  $.  This is a predicate.  It is true when $t = -100$ and false when $t = 3$.
		\item $Q(c) = P(c,c)= \textrm{``\(c^2c < 10\)''}  = \textrm{``\(c^3 < 10\)''}$. 
		
		This is a predicate.  It is true when $c = 2$, and false when $c = 3$. 
		\item $Q(a) = P(a,a)= \textrm{``\(a^2a < 10\)''}  = \textrm{``\(a^3 < 10\)''}$. 
		
		This is a predicate.  It is true when $a = 2$, and false when $a = 3$.  It might have been hard for you to substitute the second slot of $P$ with $a$, since in the initial definition we have an $a$ in the first slot, and $b$ in the second slot. You need to get used to thinking of the variables in a predicate as ``dummy variables'', which have no meaning aside from as placeholders for substitution.  Note that this predicate is functionally identical to the predicate from (5) above:  only the name of the ``dummy variable'' has changed from $c$ to $a$. 
		\item $Q(a,b) = P(b,a)  = \textrm{``\(b^2a < 10\)''}$. This may have also been challenging for the same reason.  When we write $P(a,b) = \textrm{``\(a^2b < 10\)''}$ we really mean something like $P(\square_1,\square_2) = \textrm{``\(\square_1^2\square_2 < 10\)''}$.  You can put anything you want in the boxes.  We can even put $b$ in the first box, and $a$ in the second box, although this can be confusing at first.  This predicate becomes a true proposition when we substitute $b = 1$ and $a = 5$.  It is false when $b = 2$ and $a = 5$.
	\end{enumerate}
\end{solutions}

	So far the predicates we have seen have been meaningful statements like ``$x$ is larger than $y$'', ``$s$ is the sister of $t$'', ``$2x+3y = z$'',  or ``There is a person who is taller than $x$''.

Consider the following ``mystery'' predicate.  We do not have access to the ``meaning'' of the predicate, but we have a table for looking up the truth value of the predicate given different inputs from the set $\{1,2,3,4,5\}$.

\begin{table}[h!]
	\begin{center}
		\caption{A mystery predicate.}
		\begin{tabular}{c|c} % <-- Alignments: 1st column left, 2nd middle and 3rd right, with vertical lines in between
			
			$x$ & $P(x)$ \\
			\hline
			1 & F \\ 
			2 & F \\
			3 & F \\
			4 & T \\
			5 & T 
		\end{tabular}
	\end{center}
\end{table}

There are many predicates which could generate this same table.  For example we could have

\begin{itemize}
	\item \(P_1(x) = \textrm{`` $x \geq 4$''}\)\\
	\item $P_2(x)=$ ``$x$ is either $4$ or $5$''.\\
	\item $P_3(x) =$ ``$x$ is a root of the polynomial $x^2 - 9x+20$''.\\
\end{itemize}

\begin{xca}
	Come up with 3 statements which all fit the following table for a predicate $Q(x)$:
	
	\begin{table}[h!]
		\begin{center}
			\caption{A mystery predicate.}
			\label{tab:table1}
			\begin{tabular}{c|c} % <-- Alignments: 1st column left, 2nd middle and 3rd right, with vertical lines in between
				
				$x$ & $Q(x)$ \\
				\hline
				1 & $\F$ \\ 
				2 & $\T$ \\
				3 & $\T$ \\
				4 & $\F$ \\
				5 & $\T$ 
			\end{tabular}
		\end{center}
	\end{table}
	
\end{xca}


\begin{solutions}
	Many solutions are possible.  Here are a few:
	
	\begin{enumerate}
		\item ``$x$ is a prime number''.
		\item ``$x$ is a number other than $1$ or $4$''.
		\item ``$x$ is either $2$, $3$, $5$, or $20$''.
		\item ``$x$ is not a solution to $x^2 - 5x = -4$''.
	\end{enumerate}
\end{solutions}

We can also use tables to record the truth values of $2$ variable predicates.  

\begin{xca}
	Here is a partially filled out table for the predicate
	
	\[P(x,y) = \textrm{``$xy + 1 \geq  5''$}\]
	
	Complete the rest of the table.
	
	\begin{table}[h!]
		\begin{center}
			\label{tab:table1}
			\begin{tabular}{c|c|c|c|}
				$ P(x,y)$        &$x=1$ & $x=2$ & $x=3$  \\
				\hline
				$y=1$ &    \hphantom{F}        &    \hphantom{F}        &      \hphantom{F}        \\
				\hline
				$y=2$ &    $\F$      &     $\T$      & $\T$        \\
				\hline
				$y=3$ &      \hphantom{F}      &      \hphantom{F}       &        \hphantom{F}      \\
				\hline
			\end{tabular}
		\end{center}
	\end{table}
	
\end{xca}


\begin{solutions}
	
	\begin{table}[h!]
		\begin{center}
			\label{tab:table1}
			\begin{tabular}{c|c|c|c|}
				$ P(x,y)$        &$x=1$ & $x=2$ & $x=3$  \\
				\hline
				$y=1$ &    $\F$      &     $\F$      & $\F$        \\
				\hline
				$y=2$ &    $\F$      &     $\T$      & $\T$        \\
				\hline
				$y=3$ &    $\F$      &     $\T$      & $\T$        \\
				\hline
			\end{tabular}
		\end{center}
	\end{table}
\end{solutions}

We will find that these tables of ``input/output'' values for predicates are all we need (in principle) to be able to make arguments involving predicates.  The table above is definitely more abstract than the meaningful statement ``$xy + 1 \geq  5$'', but it carries all of the information needed to make arguments about this predicate, at least when $x$ and $y$ are restricted to only take values from the set $\{1,2,3\}$.

\begin{xca}
	Given the predicate $P$ with table
	
		\begin{table}[h!]
		\begin{center}
			\begin{tabular}{c|c|c|c|}
				$ P(x,y)$        &$x=1$ & $x=2$ & $x=3$  \\
				\hline
				$y=1$ &    $\T$      &     $\F$      & $\F$        \\
				\hline
				$y=2$ &    $\F$      &     $\T$      & $\T$        \\
				\hline
				$y=3$ &    $\T$      &     $\F$      & $\T$        \\
				\hline
			\end{tabular}
		\end{center}
	\end{table}

	Make tables for each of the following predicates
	
	\begin{enumerate}
			\item $Q(a) = P(a,2)$
			\item $Q(t) = P(2,t)$
			\item $Q(t) = P(t,t)$
			\item $Q(x) = P(2,3)$
			\item $Q(a,b) = P(b,a)$
			\item $Q(a,b) = P(a,a)$
		\end{enumerate}

	\end{xca}

\begin{solutions}
			\begin{enumerate}
			\item $Q(a) = P(a,2)$
				
						\begin{table}[h!]
						\begin{center}
							\begin{tabular}{c|c}
								
								$a$ & $Q(a)$ \\
								\hline
								1 & $P(1,2) = \F$ \\ 
								2 & $P(2,2) = \T$ \\
								3 & $P(3,2) = \T$ \\
							\end{tabular}
						\end{center}
					\end{table}
			\item $Q(t) = P(2,t)$
			
									\begin{table}[h!]
				\begin{center}
					\begin{tabular}{c|c}
						
						$t$ & $Q(t)$ \\
						\hline
						1 & $P(2,1) = \F$ \\ 
						2 & $P(2,2) = \T$ \\
						3 & $P(2,3) = \F$ \\
					\end{tabular}
				\end{center}
			\end{table}
		
			\item $Q(t) = P(t,t)$
			
									\begin{table}[h!]
				\begin{center}
					\begin{tabular}{c|c}
						
						$t$ & $Q(t)$ \\
						\hline
						1 & $P(1,1) = \T$ \\ 
						2 & $P(2,2) = \T$ \\
						3 & $P(3,3) = \T$ \\
					\end{tabular}
				\end{center}
			\end{table}
		
			\item $Q(x) = P(2,3)$
			
									\begin{table}[h!]
				\begin{center}
					\begin{tabular}{c|c}
						
						$x$ & $Q(x)$ \\
						\hline
						1 & $P(2,3) = \F$ \\ 
						2 & $P(2,3) = \F$ \\
						3 & $P(2,3) = \F$ \\
					\end{tabular}
				\end{center}
			\end{table}
		
			\item $Q(a,b) = P(b,a)$
			
					\begin{table}[h!]
				\begin{center}
					\begin{tabular}{c|c|c|c|}
						$ Q(a,b)$        &$a=1$ & $a=2$ & $a=3$  \\
						\hline
						$b=1$ &    $P(1,1) = \T$      &     $P(1,2) = \F$      & $P(1,3) = \T$        \\
						\hline
						$b=2$ &    $P(2,1) = \F$      &     $P(2,2) = \T$      & $P(2,3) = \F$        \\
						\hline
						$b=3$ &    $P(3,1) = \F$      &     $P(3,2) = \T$      & $P(3,3) = \T$        \\
						\hline
					\end{tabular}
				\end{center}
			\end{table}
			
			
			\item $Q(a,b) = P(a,a)$
			
								\begin{table}[h!]
				\begin{center}
					\begin{tabular}{c|c|c|c|}
						$ Q(a,b)$        &$a=1$ & $a=2$ & $a=3$  \\
						\hline
						$b=1$ &    $P(1,2) = \T$      &     $P(2,2) = \T$      & $P(3,2) = \T$        \\
						\hline
						$b=2$ &    $P(1,2) = \T$      &     $P(2,2) = \T$      & $P(3,2) = \T$        \\
						\hline
						$b=3$ &    $P(1,2) = \T$      &     $P(2,2) = \T$      & $P(3,2) = \T$        \\
						\hline
					\end{tabular}
				\end{center}
			\end{table}
			
			
		\end{enumerate}
	\end{solutions}

\section{Quantifiers}


\begin{xca}

Consider the predicate (defined over the real numbers)

\[
A(x,y) = \textrm{``xy = 10y''}
\]

Things are going to get a little weird!  We are going to have some sentences which talk about $A(x,y)$.  Some of these sentences will be predicates, and some will not.  These will be ``meta-level'' predicates.  Decide which of the following are predicates and which are propositions. Identify what the variables of the predicates are, and see if you can find any assignments for their variables which make the predicate true or false.  Decide the truth value of the propositions.

\begin{enumerate}
		\item ``There real numbers $x$ and $y$ so that $A(x,y)$ is true.''
		\item ``$A(x,y)$ is always true, no matter what values are substituted for the variables.''
	    \item  ``For every value of  $y$,  $A(x,y)$ is true.''
		\item ``There is a value of $x$ so that $A(x,y)$ is true.''
		\item ``There is a value of $x$ so that $A(x,y)$ would be true no matter what $y$ is.''
		\item ``There is a value of $t$ so that $A(t,t)$ is true.''
	\end{enumerate}
\end{xca}

\begin{solutions}
		\begin{enumerate}
			\item ``There real numbers $x$ and $y$ so that $A(x,y)$ is true.'' - This is a true proposition.  When $x = 10$ and $y = 4$, $A(10,4)$ is the true statement that $10 \cdot 4 = 10 \cdot 4$.  Note that, although this sentence has variables, it is still a proposition:  we could decide its truth value.
			\item ``$A(x,y)$ is always true, no matter what values are substituted for the variables.'' - This is a false proposition.  When $x = 1$ and $y = 1$, we have $1 \cdot 1 = 10 \cdot 1$, which is false.  So it isn't true that $A(x,y)$ is true for all values of $x$ and $y$.
			\item  ``For every value of  $y$,  $A(x,y)$ is true.'' - This is a predicate with $x$ as a variable.  
			
			When $x = 10$, we have ``For every value of  $y$,  $A(10,y)$ is true.'', which says that $10y = 10y$ for every value of $y$.  This is a true statement.
			
			When $x = 1$, we have ``For every value of  $y$,  $A(1,y)$ is true.'', which says that $1y = 10y$ for every value of $y$.  This is a false statement.  Choosing $y = 5$ yields a false statement.
			\item ``There is a value of $x$ so that $A(x,y)$ is true.'' - This is a predicate with $y$ as a variable.  When $y = 3$ it is true, since setting $x = 10$ makes $x \cdot 3 = 10 \cdot 3$ true.  Actually, this is true no matter what $y$ is, since we could always choose $x=10$.  It is interesting to note that it is still true (but for potentially different reasons) when $y = 0$.
			\item ``There is a value of $x$ so that $A(x,y)$ would be true no matter what $y$ is.'' - This is a true proposition, as noted above. 
			\item ``There is a value of $t$ so that $A(t,t)$ is true.''  This is a true proposition.  When $t = 0$ we obtain the true statement $0 \cdot 0 = 10 \cdot 0$, for instance.
		\end{enumerate}
	\end{solutions}

Each of these is an example of a \textbf{quantified} statement.  When sentence asserts something about the number of variable assignments which make a predicate true, that is quantification.  For instance saying ``There is at least one value of $x$ which makes the proposition true'', or ``Every value of $x$ makes the proposition true'', or ``There are exactly two values of $y$ which make the proposition true''.  We will focus on only two types of quantification, universal and existential quantification, since all other quantifiers can be ``built'' out of these two quantifiers together with the logical connectives we will study in the next chapter.

When we make a new predicate $Q$ which declares that a predicate $P$ is true for all values of a given variable we are \textbf{universally quantifying} $P$ in that variable.   When we make a new predicate $Q$ which declares that a predicate $P$ is true for at least one value of a given variable we are \textbf{existentially quantifying} the predicate $P$ in that variable.  In the first problem above we have existentially quantified the predicate in both variables, while in the second problem we have universally quantified the predicate in both variables.  When a variable has been quantified we say that it is \index{bound variable}\textbf{bound} by the quantifier.  If a variable in a predicate has not been bound by a quantifier then we say the variable is \index{free variable}\textbf{free}.

\begin{definition}
	Let $P$ be a predicate with $n$ free variables $x_1,x_2,x_3, \dots, x_n$.  
	Let $x_i$ be one of the variables. Let $\mathcal{U}_i$ be the set of all allowable values for $x_i$.
		\begin{itemize} 
			\item The new predicate $Q(x_1,x_2, x_3, ..., x_{i-1}, x_{i+1}, \dots , x_n) = \forall x_i  : P(x_1,x_2,x_3, \dots, x_n)$ makes the claim:
			
			``$P(x_1,x_2,x_3, \dots, x_n)$ is true no matter what element of $\mathcal{U}_i$ is substituted in for $x_i$''.
			
			The symbol $\forall$ is called the \index{Universal Quantifier}\textbf{universal quantifier}.
			
			\item The new predicate $Q(x_1,x_2, x_3, ..., x_{i-1}, x_{i+1}, \dots , x_n) = \exists x_i  : P(x_1,x_2,x_3, \dots, x_n)$ makes the claim:

``We can find at least one element of $a \in \mathcal{U}_i$ which makes $P(x_1,x_2,x_3, \dots, x_n)$ true when we substitute $x_i = a$ .

The symbol $\exists$ is called the \index{Existential Quantifier}\textbf{existential quantifier}.

The element $a$ making the predicate true is called a \index{Witness}\textbf{witness} for the existential quantifier. 
		\end{itemize}  
	
	\end{definition}

\begin{xca}
	For each of the following sentences, decide what the free variables are.  If the sentence has no free variables, it is a proposition.  In this case, decide the truth value of the proposition.  Otherwise, it is a predicate.  Try to find some values of the variables which make the predicate true, and some which make the predicate false (if possible).
	
	\begin{enumerate}
			\item $\forall x \in \mathbb{R}: x + x = 2x$
			\item $\exists x \in \mathbb{R}: x + x = 2x$
			\item $\forall x \in \mathbb{R}: 2x^2 + 3 = 11$
			\item $\exists x \in \mathbb{R}: 2x^2 + 3 = 11$
			\item $\forall t \in \mathbb{R}: t^2 > 0$
			\item $\exists t \in \mathbb{R}: t^2 > 0$
			\item $\exists t \in \mathbb{R}: t^2 \leq 0$
			\item $\forall y \in \mathbb{R}:  xy = 0$
			\item $\exists y \in \mathbb{R}:  xy = 1$
			\item $\forall x \in [0,1]:  x \leq y$
			\item $\exists x \in [0,1]: x \leq y$
		\end{enumerate}
	\end{xca}

\begin{solutions}
		\begin{enumerate}
		\item $\forall x \in \mathbb{R}: x + x = 2x$ - This has no free variables, and is a proposition.  It is true!  No matter what real number $x$ we substitute into the predicate ``$x+x  =2x$'' , we obtain a true proposition.
		\item $\forall x \in \mathbb{R}: 2x^2 + 3 = 11$ - This has no free variables, and is a proposition.  It is false!  If we substitute $x = 0$ into the predicate ``$2x^2+3 = 11$'' we obtain the false statement that $2(0)^2+3 = 11$.  So the predicate $2x^2+3 = 11$ is not true for every value of $x$.
		\item $\exists x \in \mathbb{R}: 2x^2 + 3 = 11$ - This has no free variables, and is a proposition.  It is true!  If we substitute $x = -2$ (for instance) into the predicate ``$2x^2+3 = 11$''  we obtain the true statement that $2(-2)^2 + 3 = 11$.
		\item $\forall t \in \mathbb{R}: t^2 > 0$ - This statement has no free variables, and is a proposition.  It is false.  If we substitute $t=0$ into the predicate ``$t^2>0$'', we obtain the false proposition that $0>0$.
		\item $\exists t \in \mathbb{R}: t^2 > 0$ - This statement has no free variables, and is a proposition.  It is true.  If we substitute $t=1$ (for instance) into the predicate ``$t^2>0$'', we obtain the true proposition that $1>0$.
		\item $\exists t \in \mathbb{R}: t^2 \leq 0$ - This statement has no free variables, and is a proposition.  It is true.  If we substitute any real number $t$ into the predicate ``$t^2 \geq0$'', we obtain the a true proposition.
		\item $\forall y \in \mathbb{R}:  xy = 0$ - This statement has $x$ as a free variable, and so is a  predicate.  When we substitute $x = 0$, we obtain the true statement that ``For all real numbers $y$, $0 \cdot y = 0$''.  When $x = 1$ we obtain the false statement that ``For all real numbers $y$, $1 \cdot y = 0$''.
		\item $\exists y \in \mathbb{R}:  xy = 1$ - This statement has $x$ as a free variable, and so is a  predicate.  When we substitute $x = 5$, we obtain the true statement that ``There is a real number $y$ with  $5 \cdot y = 1$'' (true, since $y = \frac{1}{5}$ works).   When $x = 0$ we obtain the false statement that ``There is a real number $y$ with  $0 \cdot y = 1$'' (false, since any value for $y$ would yield the false statement $ 0 = 1$).
		\item $\forall x \in [0,1]:  x \leq y$ - This statement has $y$ as a free variable, and so is a predicate.  When we substitute $y=1$ into the proposition `` $\forall x \in [0,1]:  x \leq y$'' we obtain the true proposition `` $\forall x \in [0,1]:  x \leq 1$''.  When choose  $y = 0.5$ we obtain the false proposition `` $\forall x \in [0,1]:  x \leq 0.5$''. 
		\item $\exists x \in [0,1]: x \leq y$  - This statement has $y$ as a free variable, and so is a predicate.  When we substitute $y=0.5$ into the proposition `` $\exists x \in [0,1]:  x \leq y$'' we obtain the true proposition `` $\exists x \in [0,1]:  x \leq 0.5$''. This is true since, for instance, $0.3 < 0.5$.  When choose  $y = -1$ we obtain the false proposition `` $\exists x \in [0,1]:  x \leq -1$''.  Any substitution of $x$ from the set of allowable values would yield a false proposition in this case. 
	\end{enumerate}
	\end{solutions}

	
	\begin{xca}
		
		Given the following table of values for the predicate $P$, decide whether the propositions below are true or false.  Explain your reasoning.
		
						\begin{table}[h!]
			\begin{center}
				\label{tab:table1}
				\begin{tabular}{c|c|c|c|}
					$P(x,y)$ &$x=3$ & $x=4$ & $x=5$  \\
					\hline
					$y=1$    &   $\F$        &     $\T$       &     $\T$      \\
					\hline
					$y=2$   &    $\F$       &     $\T$        &    $\T$      \\
					\hline
					$y=3$   &    $\F$       &      $\F$        &   $\T$       \\
					\hline
				\end{tabular}
			\end{center}
		\end{table}
	
				\begin{enumerate}
		\item $P(4,1)$
		\item $P(4,2)$
		\item $\exists y \in \{1,2,3\}: P(3,y)$
		\item $\exists y  \in \{1,2,3\}: P(4,y)$
		\item $\forall y  \in \{1,2,3\}: P(4,y)$
		\item $\forall y  \in \{1,2,3\}: P(5,y)$
		\item $\forall x  \in \{3,4,5\}: P(x, 1)$
		\item $\exists x  \in \{3,4,5\}: P(x,1)$
		\item $\exists b  \in \{3,4,5\}: P(b, 3)$
		\item $\exists t  \in \{3,4,5\}: P(t, t-2)$
		\item $\exists y  \in \{3,4,5\}: P(y,1)$
	\end{enumerate}
		

	\end{xca}

\begin{solutions}



		\begin{enumerate}
	\item $P(4,1) = \T$
	\item $P(4,2) = \T $
	\item $\exists y \in \{1,2,3\}: P(3,y) = \F$.  $P(3,1)$, $P(3,2)$, and $P(3,3)$ are all false.  So there is no value of $y$ from the set $\{1,2,3\}$ which makes $P(3,y)$ true.
	\item $\exists y  \in \{1,2,3\}: P(4,y) = \T$.  Setting $y=2$ gives $P(4,2) = \T$, so there is at least one value of $y$ from the set $\{1,2,3\}$ which makes $P(4,y)$ true.
	\item $\forall y  \in \{1,2,3\}: P(4,y) = \F$. Setting  $y = 3$ gives $P(4,3) = \F$, so $P(4,y)$ is not true for every value of $y$ in $\{1,2,3\}$ (even though it is true for some of them).
	\item $\forall y  \in \{1,2,3\}: P(5,y) = \T$.  $P(5,1)$, $P(5,2)$, and $P(5,3)$ are all true.  So the statement that $P(5,y)$ is true for every $y$ in $\{1,2,3\}$ is true.
	\item $\forall x  \in \{3,4,5\}: P(x, 1) = \F$.  When $x = 3$, $P(3, 1) = \F$, so it isn't true that $P(x,1)$ is true for all values of $x$ in \{1,2,3\}.
	\item $\exists x  \in \{3,4,5\}: P(x,1) = \T$.  $P(4,1) = \T$ for instance.
	\item $\exists b  \in \{3,4,5\}: P(b, 3) = \T$.  When $b = 5$, we obtain the true proposition $P(5,3)$.
	\item $\exists t  \in \{3,4,5\}: P(t, t-2) = \T$.  When $t = 4$, we obtain the true proposition $P(4,2)$.  
	\item $\exists y  \in \{3,4,5\}: P(y,1) = \T$.  Remember not to get caught up in variable names here!  Despite being called $y$, this variable is in the first slot.  A value of $y$ which works is $y = 4$, since $P(4,1) = \T$. 
\end{enumerate}
	
	\end{solutions}

\section{Nested Quantifiers}

If we start with a predicate with $n$ free variables and quantify one of the variables we obtain a new predicate with $n-1$ free variables.  There is nothing stopping us from quantifying one of the free variables in this new predicate!  When we have a quantifier applying to a predicate which is itself quantified, we call this \index{Nested Quantifiers} \textbf{Nested Quantification}.

Consider the following proposition:

\begin{centering}
		``There is a person who is as rich or richer than everyone.''
\end{centering}

Let $R(x,y)$ be the predicate ``$x$ is as rich or  richer than $y$'', where $x$ and $y$ are allowed to be people.

Then the predicate $\forall y: R(x,y)$ is the predicate ``$x$ is as rich or richer than everyone''.

So the original proposition could be thought of as $\exists x: \left( \forall y: R(x,y) \right)$.

When we have a predicate with $n$ variables, we can quantify each variable to obtain a proposition.

\begin{xca}
Let $R(x,y)$ be the predicate ``$x$ is as rich or  richer than $y$'', where $x$ and $y$ are allowed to be people.

Translate each of the following into English sentences which have the same meaning.  Decide whether they are true or false.

\begin{enumerate}
		\item $R(\textrm{Lebron James}, \textrm{PeeWee Herman})$
		\item $\exists x: R(x,\textrm{Lebron James})$
		\item $\forall t: R(\textrm{Jeff Bezos}, t )$
		\item $\forall t: R(t,t)$
		\item $\exists x: \left( \forall y : R(x,y)\right)$
		\item $\forall y : \left( \exists x : R(x,y)\right)$
		\item $\forall x : \left( \exists y : R(x,y)\right)$
		\item $\exists y : \left( \forall  x : R(x,y)\right)$
	\end{enumerate}
	\end{xca}

\begin{solutions}
	
	All of these solutions are accurate as of 05/17/2021, but fortunes are subject to change.
	\begin{enumerate}
		\item $R(\textrm{Lebron James}, \textrm{PeeWee Herman})$ - This proposition claims that Lebron James is as rich or richer than PeeWee Herman, which can be verified as true through an internet search.
		\item $\exists x: R(x,\textrm{Lebron James})$ - This proposition claims that someone is as rich or richer than Lebron James.  This is true.  Bill Gates is richer than Lebron James (at the time of this writing).  A more canonical choice here might be to let $x$ be Lebron James!  Certainly Lebron James is as rich or richer than Lebron James.  This also verifies the proposition.
		\item $\forall t: R(\textrm{Jeff Bezos}, t )$ - This proposition claims that Jeff Bezos is as rich or richer than everyone.  This is true.
		\item $\forall t: R(t,t)$ -  This proposition claims that everyone is as rich or richer than themselves, which is true.
		\item $\exists x: \left( \forall y : R(x,y)\right)$ - This proposition claims that someone is as rich or richer than everyone.  This is true, one choice that works for $x$ is Jeff Bezos.
		\item $\forall y : \left( \exists x : R(x,y)\right)$  - This proposition claims that no matter who $y$ is, you can always find someone ($x$) who is as rich or richer $y$.  This is true, since we could just take $x = y$.  Everyone is as rich or richer than themselves.
		\item $\forall x : \left( \exists y : R(x,y)\right)$ - This proposition claims that no matter who $x$ is, you can find someone ($y$) so that $x$ is as rich or richer than $y$.  This is also true, since we can take $y = x$.
		\item $\exists y : \left( \forall  x : R(x,y)\right)$ - This proposition claims that there is a person $y$ so that anyone else $x$ is as rich or richer than $y$ is.  This is true:  let $y$ be the poorest person alive (if multiple people are tied, we can choose any one of them to be $y$).
	\end{enumerate}
	\end{solutions}


\begin{xca}
	Let $P(x,y)$ be the predicate ``$y >  x^2$''. Let $\mathbb{R}^+$ be the set of all non-negative real numbers (includes $0$ and all positive numbers, but no negative numbers).  Which of the following propositions are true and which are false?  Explain your reasoning.
	
	\begin{enumerate}
			\item $\exists t \in \mathbb{R}^+: P(t,t)$
			\item $\forall t \in \mathbb{R}^+: P(t,t)$
			\item$\exists x \in \mathbb{R}^+: \exists y\in \mathbb{R}^+: P(x,y)$
			\item $\exists x \in \mathbb{R}^+: \forall y \in \mathbb{R}^+: P(x,y)$
			\item $\forall y \in \mathbb{R}^+: \exists x \in \mathbb{R}^+: P(x,y)$
			\item $\exists y \in \mathbb{R}^+: \forall x \in \mathbb{R}^+: P(x,y)$
			\item $\forall x \in \mathbb{R}^+: \exists y \in \mathbb{R}^+: P(x,y)$
		\end{enumerate}
	\end{xca}

\begin{solutions}
			\begin{enumerate}
			\item $\exists t \in \mathbb{R}^+: P(t,t)$ - This is true.   $t = \frac{1}{2}$ is a witness since $\frac{1}{2} > (\frac{1}{2})^2  =\frac{1}{4}$.
			\item $\forall t \in \mathbb{R}^+: P(t,t)$ - This is false.  For instance, when $ t= 1$, $P(1,1) = $ ``$1 > 1^2$'', which is false.  
			\item$\exists x \in \mathbb{R}^+: \exists y\in \mathbb{R}^+: P(x,y)$ - This is true.  When $x = 1$, we are trying to find a $y$ so that $y > 1^2$.  $y = 2$ yields a true statement.
			\item $\exists x \in \mathbb{R}^+: \forall y \in \mathbb{R}^+: P(x,y)$ - This is false.  No matter what $x$ is, $x^2$ will be greater or equal to $0$.  So it cannot be true that all $y \in \mathbb{R}^+$ are greater than $x^2$, since $y = 0$ is not strictly greater than $0$.
			\item $\forall y \in \mathbb{R}^+: \exists x \in \mathbb{R}^+: P(x,y)$ - This is false.  Taking $y=0$, there is no $x$ for which $0>x^2$.
			\item $\exists y \in \mathbb{R}^+: \forall x \in \mathbb{R}^+: P(x,y)$ - This is false.  No matter what $y$ is, if we choose $x$ to be larger than $\sqrt{y}$, we would have $x^2 > y$.
			\item $\forall x \in \mathbb{R}^+: \exists y \in \mathbb{R}^+: P(x,y)$ - This is true.  No matter what $x$ is, we could choose $y = x^2+1$ and $P(x,y)$ would be $P(x,x^2+1)$ which is the true statement ``$x^2+1 > x^2$''.
		\end{enumerate}
	\end{solutions}


	\begin{xca}
	
	Given the following table of values for the predicate $P$, decide whether the propositions below are true or false.  Explain your reasoning.
	
	\begin{table}[h!]
		\begin{center}
			\label{tab:table1}
			\begin{tabular}{c|c|c|c|}
				$P(x,y)$ &$x=1$ & $x=2$ & $x=3$  \\
				\hline
				$y=1$    &   $\F$        &     $\T$       &     $\T$      \\
				\hline
				$y=2$   &    $\F$       &     $\T$        &    $\T$      \\
				\hline
				$y=3$   &    $\F$       &      $\T$        &   $\F$       \\
				\hline
			\end{tabular}
		\end{center}
	\end{table}
	
	\begin{enumerate}
		\item $\exists t: P(t,t)$
		\item $\forall t: P(t,t)$
		\item $\exists a: \forall b: P(a,b)$
		\item $\exists a: \forall b: P(b,a)$
		\item $\forall t: \exists s: P(t,s)$
		\item $\forall t: \exists s: P(s,t)$
	\end{enumerate}
	
	
\end{xca}

\begin{solutions}
		\begin{enumerate}
		\item $\exists t: P(t,t)$ - This is a true.  $t = 2$ is a witness since $P(2,2) = \T$
		\item $\forall t: P(t,t)$ - This is false.  For instance, setting $t =1$, $P(1,1) = \F$.
		\item $\exists a: \forall b: P(a,b)$ - This is true.  When $a = 1$, we have the statement $\forall b: P(1,b)$.  This is a true statement since $P(1,1)$, $P(1,2)$, and $P(1,3)$ are all true.
		\item $\exists a: \forall b: P(b,a)$ - This is a false statement.  
		\begin{itemize}
				\item If we try $a = 1$, we have the statement $\forall b: P(b,1)$.  However, $P(1,1)$ is false.
				\item If we try $a = 2$, we have the statement $\forall b: P(b,2)$.  However, $P(1,2)$ is false.
				\item If we try $a = 3$, we have the statement $\forall b: P(b,3)$.  However, $P(3,3)$ is false.
			\end{itemize}
		
		So there is no value of $a$ which makes the predicate $\forall b: P(b,a)$ true.
		\item $\forall t: \exists s: P(t,s)$ - This is a false statement.  When $t  =1$, we have the proposition $\exists s: P(1,s)$.  This is false since $P(1,1)$, $P(1,2)$, and $P(1,3)$ are all false.
		\item $\forall t: \exists s: P(s,t)$  - This is a true statement.  
		
				\begin{itemize}
			\item If we try $t = 1$, we have the statement $\exists s: P(s,1)$.  There is such an $s$, namely $s = 2$ or $s = 3$.
			\item If we try $t = 2$, we have the statement $\exists s: P(s,2)$.  There is such an $s$, namely $s = 2$ or $s = 3$
			\item If we try $t = 3$, we have the statement $\exists s: P(s,3)$.  There is such an $s$, namely $s = 2$.
		\end{itemize}
	\end{enumerate}
	\end{solutions}


\section{A sampling of definitions involving quantifiers}

Now that we understand propositions, predicates, and quantifiers, we are equipped to make some of our first rigorous mathematical definitions.

\subsection{Parity}
You are probably already familiar with the notion of even and odd numbers.   The even numbers are 

$$
\dots, -6, -4, -2, 0 , 2, 4, 6,  \dots
$$

and the odd numbers are 

$$
\dots,  -5, -3, -1, 1, 3, 5, \dots
$$

The \index{parity}\textbf{parity} of an integer is whether it is even or odd.  The parity of 8 is even, while the parity of 9 is odd.

If we want to reason about the parity of integers, we will need precise definitions:

\medskip

\begin{definition}[Even number]
	
	An integer is called \index{Even Integer}\textbf{even} if it is twice an integer. Symbolically, we say that $n$ is even if $$\exists k \in \Z: n=2k$$
\end{definition}

\begin{definition}[Odd number]
	
	An integer is called \index{Odd Integer}\textbf{odd} if it is one more than twice an integer. Symbolically, we say that $n$ is odd if $$\exists k \in \Z: n=2k+1$$
\end{definition}

It is important to realize that, in mathematics, we have to be very careful to argue exclusively from our definitions.  You might know many other things about even and odd numbers.  Since you have this background knowledge, you might be tempted to give arguments like ``$13$ is odd because its last digit is odd'' or ``$13$ is odd because it isn't even''.  Those are both valid reasons, but they are not justifications which follow from the definitions.

In mathematics, we start with our definitions and carefully prove a web of interconnected theorems about the objects we are studying.  So the justification ``$13$ is odd because its last digit is odd'' is only valid after we have already proved the theorem ``A number is odd if its last digit is odd'' and the theorem that ``$13$ is an odd number''.

In the famous words of Humpty Dumpty \cite{SV}

\begin{quote}
	``When \textit{I} use a word," Humpty Dumpty said, in rather a scornful tone, ``it means just what I choose it to mean - neither more nor less."
\end{quote}

We must also be this careful. When we give a definition, that is what the word means:  neither more nor less.  The word might have more meanings for you in a different context, but in the context of the mathematical work we are doing we must practice good logical hygiene.

\begin{xca}
	Argue explicitly from the definitions that each of the following statements is true:
	
	\begin{enumerate}
			\item ``$6$ is even.''
			\item ``$11$ is odd.''
			\item ``$0$ is even.''
			\item ``$1$ is odd.''
			\item ``$-10$ is even.''
			\item ``$-7$ is odd.''
		\end{enumerate}
	\end{xca}

\begin{solutions}
	
	\begin{enumerate}
		\item ``$6$ is even.'' - we need to show that there is at least one integer $k$ with $6 = 2k$.  When $k = 3$, we do have $6 = 2(3)$.  Thus $6$ is even.
		\item ``$11$ is odd.''- we need to show that there is at least one integer $k$ with $11 = 2k+1$.  When $k = 5$, we do have $11 = 2(5)+1$.  Thus $11$ is odd.
		\item ``$0$ is even.''- we need to show that there is at least one integer $k$ with $0 = 2k$.  When $k = 0$, we do have $0 = 2(0)$.  Thus $0$ is even.
		\item ``$1$ is odd.'' - we need to show that there is at least one integer $k$ with $1 = 2k+1$.  When $k = 0$, we do have $1 = 2(0)+1$.  Thus $1$ is odd.
		\item ``$-10$ is even.''- we need to show that there is at least one integer $k$ with $-10 = 2k$.  When $k = -5$, we do have $6 = 2(-5)$.  Thus $-10$ is even.
		\item ``$-7$ is odd.''- we need to show that there is at least one integer $k$ with $-7 = 2k+1$.  When $k = -4$, we do have $-7 = 2(-4)+1$.  Thus $-7$ is odd.
	\end{enumerate}

\end{solutions}

\subsection{Divisibility}

You probably have some prior knowledge of what it means for one integer to be a \textbf{factor} of another integer.  $3$ is a factor of $15$ because $15 = 3(5)$, and $10$ is a factor of $-60$ because $-60 = 10(-6)$.  

You may be less familiar with the word \textbf{divides}.  Intuitively, $a$ divides $b$ if $a$ is a factor of $b$.  So we could rephrase the above examples by saying $3$ divides $15$ and $10$ divides $-60$.  You might be more comfortable if you mentally insert the words ``into'' and ``evenly'' when you read these.  Read:  ``$3$ divides (into) $15$ (evenly)''.  Although these two extra words do clarify what is meant, it is the habit of mathematicians to omit them when speaking and writing, so you should also practice this convention if you want to speak with mathematicians.

In fact, there are a large number of different phrases expressing the same concept.  We will consider all of the following phrases to be equivalent:

\begin{itemize}
	\item $a$ divides $b$.
	\item $a$ is a factor of $b$.
	\item $b$ is a multiple of $a$.
	\item $b$ is divisible by $a$.
	\item Symbolically we will write $a \divides b$ for all of these equivalent statements.  It is typical to read this as ``$a$ divides $b$'', but you could also read it as any of the other statements above. This symbol is designed to be similar to the inequality symbol, since when both $a$ and $b$ are positive  if $a \divides b$ then $a \leq b$.  This should help you remember that $4 \divides 12$ and not the other way around.  Our use of $\divides$ is not standard:  most authors use the symbol $|$ where we use $\divides$.  The more standard symbol $|$ is visually symmetric, and this causes students to frequently forget which order is intended.  I hope the choice of $\divides$ for this symbol is helpful to you.
\end{itemize}

We can generalize from these examples to give a formal definition:

\medskip

\begin{definition}[Official Definition]
	An integer $a$ is said to \index{divide}\textbf{divide} an integer $b$ if there is an integer $k$ such that
	
	$$
	b = ak
	$$
	
	Symbolically we write  $a \divides b$ if 
	
	$$
	\exists k \in \Z: b=ak
	$$
\end{definition}

When you first meet a definition, you should find some examples, non-examples, and also explore any potential ``pathological'' or ``unintuitive'' examples.  

\begin{xca}
	\begin{enumerate}
	\item Does $3$ divide $12$? 
	\item Does $12$ divide $3$?  
	\item Does $7$ divide $7$? 
	\item Does $(-5)$ divide  $5$  
	\end{enumerate}
\end{xca}

\begin{solutions}
			\begin{enumerate}
			\item Does $3$ divide $12$?  - Yes!  There is an integer $k$ with $12 = 3k$, namely $k=4$. 
			\item Does $12$ divide $3$?  - No.  There is no integer $k$ for which $3 = 12k$.  If there were, then $k$ would have to be $\frac{1}{4}$, which is not an integer.
			\item Does $7$ divide $7$?  - Yes!  There is an integer $k$ with $7 = 7k$, namely $k=1$. 
			\item Does $(-5)$ divide  $5$  - Yes!  There is an integer $k$ with $5 = (-5)k$, namely $k=-1$.
		\end{enumerate}
	\end{solutions}

When answering these questions it is important that you use our official definition.  You may have another definition of divisibility which you have learned from another instructor, or just picked up on intuitively.  Many students have the following unofficial definition in mind when they think about divisibility:

\medskip

\begin{definition}[Warning:  Unofficial Definition]
	An integer $a$ is said to \textbf{divide} an integer $b$ if $\frac{b}{a}$ is an integer.
\end{definition} 

The official definition and the unofficial one agree for almost all pairs of integers $a$ and $b$.  However, the definitions are not equivalent!  They differ in how they deal with divisibility of $0$.  Explore the following questions using both definitions of divisibility to see how they are the same and how they differ.

\begin{xca}
	\begin{enumerate}
	\item Does $0$ divide $2$?  
	\item Does $2$ divide $0$?  
	\item Does $0$ divide $0$?
	\end{enumerate}
\end{xca}

\begin{solutions}
	\begin{enumerate}
		\item Does $0$ divide $2$?
		
		\begin{itemize}
			\item According to our official definition, the answer is no.  There is no integer $k$ for which $2 = 0k$.  $0k = 0$ no matter what $k$ is, and $0 \neq 2$.
			\item According to the unofficial definition, the answer is unclear.  When we try to perform $\frac{2}{0}$, the answer is not defined.  Should I count undefined as an integer or not?  Probably not, but this is a little ambiguous.
			\end{itemize}
		\item Does $2$ divide $0$? 
		
		\begin{itemize}
			\item According to our official definition, the answer is yes.  There is an integer $k$ for which $0 = 2k$, namely $k=0$.
			\item According to the unofficial definition, the answer is also yes.  $\frac{0}{2} = 0$ which is an integer.
			\end{itemize}
		\item Does $0$ divide $0$? 
				\begin{itemize}
			\item According to our official definition, the answer is yes.  There is an integer $k$ for which $0 = 0k$, such as $k=11$.
			\item According to the unofficial definition, the answer is not clear.  $\frac{0}{0}$ is also undefined.  The unofficial definition is ambiguous in this case.
		\end{itemize}
	\end{enumerate}
\end{solutions}

Since these two definitions are not equivalent it is important to only use the official definition when making arguments about divisibility!

When mathematicians encounter a new definition, we like to \textbf{play} with it.  As a budding mathematician, you should learn to do this too!

Here are some questions which immediately come up for me:

\begin{itemize}
	\item How does divisibility interact with addition?  For example, if $3$ divides two numbers, does it also divide their sum?  What about if $3$ and $5$ both divide the same number?  Does $3+5 = 8$ also have to divide that number?
	\item How does divisibility interact with multiplication?
	\item How does divisibility interact with itself?  If I know $a$ divides $b$, and $b$ divides $c$, do I know anything about the divisibility relationship between $a$ and $c$?  What if I know $a$ divides $b$ and $c$ divides $b$?  Do I know anything about the divisibility relationship between $a$ and $c$ in this case?
	\item It seems like divisibility is related to even and odd numbers somehow.  Can I make that precise?  Could I rephrase the definition of even or odd using the word ``divides'' somehow?
\end{itemize}

Can you come up with more interesting questions to play with?

One way to play with these questions is just to start making lots of examples, and seeing what happens.  You might notice a pattern.  Once you notice a pattern, you can try to state the pattern you are observing in a logically precise way.  This is a statement which you are guessing might be true.  This kind of statement is called a \textbf{conjecture}.  You can then try to prove your conjecture.  You might find that part way through your proof you get stuck.  There are a few reasons you might get stuck proving your conjecture:

\begin{itemize}
	\item The conjecture is true, but is just too hard for you to prove right now.  Maybe you or someone else can prove it later.
	\item The conjecture is false, but in a way which can be corrected.  You might discover while trying to prove the theorem that you actually need another hypothesis.  Try to find a counterexample to show that the hypothesis is necessary, and then modify your conjecture to include the hypothesis.  Now try to prove this new conjecture.
	\item The conjecture is false and there is no way to save it.  Maybe the pattern you found was based on a coincidence in the examples you chose, but doesn't actually hold in general.
\end{itemize}

To see what this looks like, let's try playing with the first question:  how does divisibility interact with addition?

Let's play:

\begin{itemize}
	\item $3 \divides 21$ and $3 \divides 15$.  Does $3 \divides (21+15)$? Yes!  $21+15 = 36 = 3(13)$, so $3 \divides (21+15)$.
	\item $7 \divides 7$ and $7 \divides 7$.  Does $7 \divides (7+7)$?   Yes!  $7+7  =14 = 7(2)$, so $7 \divides (7+7)$.
	\item Try your own examples here.  Include some weird ones, especially involving $0$.
\end{itemize}


Did your play always lead to the same conclusion?  If so we can make a conjecture:

\begin{conjecture}
	Let $a,b$ and $c$ be integers.  If $a$ divides $b$ and $a$ divides $c$, then $a$ divides $b+c$.
\end{conjecture}

We will learn how to prove conjectures like this one in the next two chapters.

%\subsection{Bounds}
%
%The three definitions we met so far all involved a single existential quantifier.  We diversify our buffet of definitions a bit by giving some definitions involving universal quantifiers and nested quantifiers.
%
%\begin{definition}
%		A real number $x$ is said to be an \index{Upper Bound}\textbf{upper bound} for a set of real numbers $S$ if $x$ is greater than or equal to every element of $S$.  Symbolically:
%		
%		\[
%		\forall y \in S: y \leq x
%		\]
%		
%		Similarly $x$ is a \index{Lower Bound}\textbf{lower bound} if
%		
%		\[
%		\forall y \in S: x \leq y
%		\]
%		
%		A set $S$ is said to be \index{bounded}\textbf{bounded} if it has an upper bound and a lower bound.
%\end{definition}
%
%\begin{xca}
%		\begin{enumerate}
%				\item Let $S = \{1,3,7,11,21\}$.  Give two different upper bounds for $S$, and justify that they are upper bounds.
%				\item Let $X = [-2,6]$.  Give two different lower bounds for $S$, and justify that they are lower bounds.
%				\item Invent your own set $S$ which has no upper bound but does have a lower bound.
%				\item Invent your own set $S$ which has no lower bound but does have an upper bound.
%				\item Does $\mathbb{N} = \{0,1,2,3,4,\dots\}$ have an upper bound? A lower bound?
%				\item Invent your own set $B$ which is bounded.
%			\end{enumerate}
%	\end{xca}
%
%\begin{solutions}
%		\begin{enumerate}
%				\item Both $21$ and $100$ are upper bounds for $S$.  We can see that each element of $S$ is less or equal to both $21$ and $100$.
%				\item Both $-2$ and $-100$ are lower bounds of $X$.  We can see that every element of $X$ is greater than or equal to  both $-2$ and $-100$.
%				\item $S = [0,\infty]$ has no upper bound, but does have $0$ as a lower bound.
%				\item $S = (-\infty,0]$ has no lower bound, but does have $0$ as an upper bound.
%				\item $\mathbb{N}$ has no upper bound, but is bounded below by $4$ (for instance).
%				\item $B = \{0,2,3\}$ is bounded, since $0$ is a lower bound and $3$ is an upper bound.
%			\end{enumerate}
%	\end{solutions}
%
%\begin{definition}
%		A set $S$ of real numbers is said to be \index{Unbounded From Above}\textbf{unbounded from above} if 
%		
%		\[
%		\forall L \in \mathbb{R}: \exists x \in S: x \geq L
%		\]
%		
%		Similarly 
%		
%				A set $S$ of real numbers is said to be \index{Unbounded From Below}\textbf{unbounded from below} if 
%		
%		\[
%		\forall L \in \mathbb{R}: \exists x \in S: x \leq L
%		\]
%	\end{definition}
%
%\begin{xca}
%		\begin{enumerate}
%				\item Argue that $\mathbb{N}$ is unbounded from above.
%				\item Argue that $\mathbb{Z}$ is unbounded from below.
%				\item Give an example of a set of real numbers which is unbounded from below, but not from above.
%			\end{enumerate}
%	\end{xca}
%
%\begin{solutions}
%	\begin{enumerate}
%		\item Given any real number $L$, we can always find a natural number $x$ which is larger than $L$.  In particular, we can just round $L+1$ up to the next integer, and take that as our choice of $X$.
%		\item Given any real number $L$, we can always find an integer $x$ which is less than $L$.  In particular, we can just round $L-1$ down to the next lowest integer, and take that as our choice of $x$.
%		\item An example of a set which is unbounded from below, but not from above, is the set \{\dots, -4,-3,-2,-1\} of negative integers. 
%	\end{enumerate}
%\end{solutions}

\section{Using and proving quantified propositions}

The primary goal of this text is to give you a precise understanding of the logic and reasoning used in mathematical arguments.  

In a mathematical argument we will often have some collection of propositions which we already know to be true, and some statements which we believe to be true and would like to prove (i.e. to argue that they are true).

In this section we will learn both how to \textbf{use} quantified statements which we already know to be true in our arguments and how to \textbf{prove} quantified statements whose truth we wish to establish.

Throughout this text we will be using ``Natural Deduction Outlines'' to help us craft our arguments.  The creation of these outlines is a somewhat involved algorithm.  In this section we will learn the parts of this algorithm which are needed for proving quantified propositions. 

\subsection{Using a universally quantified sentence}

If you know (somehow) that the sentence $\forall x \in \mathcal{U}, P(x)$ is true and if you have a particular element $a$ of your universe of discourse $\mathcal{U}$, then you also know that $P(a)$ is true.  This is sometimes called by the fancy name of ``Universal Elimination" since it is the means by which we may ``eliminate'' the universal quantifier to obtain a proposition without that quantifier which we can use in our argument.

For example, if you know that ``For every integer $x$, $6x$ is even'', and at some point in your argument you need that $6*101$ is even, you can  quote the theorem using $x = 101$ to support your argument.

\subsection{Proving a universally quantified sentence}

If your universe of discourse is finite, then you can prove $\forall x: P(x)$ by just checking each element.  If $P(x)$ is true for every single input $x$, then the universally quantified statement is true.  If there is even a single value of $x$ which makes $P(x)$ false, then $\forall x: P(x)$ is false.

However, if your universe of discourse is infinite, we cannot prove a universally quantified sentence by manually checking all of the cases.  

What we do is to choose a ``generic'' or ``arbitrary'' element of our universe of discourse, give it a name, and then argue that $P$ is true for that generic element.  This argument form sometimes goes by the fancy name ``Universal Introduction'', since it is the means by which we may ``introduce'' a universally quantified statement as an established truth in our argument.

Say you want to prove that ``Everyone who has a beard gets crumbs in it sometimes''.  This could be argued by saying ``Imagine someone who has a beard. Let's just call them Bob.  Then [arguments].  So we can conclude that Bob sometimes gets crumbs in their beard.  There was nothing special about Bob.  Therefore everyone with a beard sometimes gets crumbs in it''.

The natural deduction outline for the universally quantified sentence  $\forall x: P(x)$ is:

\begin{fitch*}
	\textrm{Let $a$ be an arbitrary element of the universe of discourse.}\\
	\textrm{Prove $P(a)$}.\\
	\textrm{Conclude $\forall x P(x)$.}
\end{fitch*}

Note:  it is important to avoid variable name conflicts!  If you have already named the variable $a$ earlier in the argument, and that name is still ``in use'', then reusing it is confusing.  Imagine you were telling a story about your friends and you said

\begin{quote}
``My one friend, lets just call them Joe, started going out with my other friend, lets just call them Keith.  What Joe didn't know is that Keith was going out with my other friend.  Let's just call this other friend Joe.  So Joe found out about Joe, and you can imagine the kind of upset that caused!''
\end{quote}

Accidentally calling two different friends by the same name has caused the potential for real confusion in your story.  Similar dangers are possible in the mathematical stories you try to tell.  Calling two different mathematical objects the same name can lead you to make logical errors.

\subsection{Using Existentially Quantified Sentences}

If you know (somehow) that the sentence $\exists x : \in \mathcal{U}, P(x)$ is true, then you know there is at least one value which makes the predicate true, but you do not know which one.  So you can choose one of the things which makes it true, and give it a name (like $a$) so you can refer to it later, but you may not make any other assumptions about the nature of $a$.  This form of argument is called ``Existential Elimination''

A typical example of this is that if you know that $a$ is odd, that means (by definition) that $\exists k \in \mathbb{Z}: a=2k+1$.  So you can choose such an element, and call it (say) $k_1$.  Then you know that $a=2k_1+1$, but you do not know anything else about $k_1$.  

Similar remarks about variable conflicts apply here:  do not choose a variable name which is already is use.  If $b$ is an even number, and you already know $a=2m+1$, do not say $b = 2m$, because you are using the same variable $m$ to reference two potentially different constants.  Instead choose a variable name you have not used yet, like $n$.  So if we know $a$ is odd and $b$ is even, we can declare integers $m$ and $n$ so that $a=2m+1$ and $b = 2n$.

\subsection{Proving Existentially Quantified Sentences}

To prove an existentially quantified sentence you need to find a candidate element $a$ of the universe of discourse for which you believe that $P(a)$ is true, and then supply a proof that $P(a)$ is actually true.  This argument form is called ``Existential Introduction''.  We only need to find a single such $a$.  A choice of $a$ is sometimes called a ``witness'' of the existentially quantified statement.

The natural deduction style proof outline for $\exists x:  P(x)$ looks like this:

\begin{fitch*}
	\textrm{(To prove $\exists x: P(x)$)}\\
	\textrm{Construct a candidate $a$ in the universe of discourse.}\\
	\textrm{Give a proof that $P(a)$ is true.}\\
	\textrm{Conclude $\exists x: P(x)$.}
\end{fitch*} 

Note:  Existentially quantified sentences are a bit funny.  The construction of a witness is often more important than the statement that the witness exists.  It is much more useful to know that $1729$ is an integer which is expressible as sum of two cubes in two different ways ($1729 = 1^3+12^3 = 9^3+10^3$ ) than it is to merely know that there is \textit{some} integer which is expressible as a sum of two cubes in two different ways. \footnote{ See \url{https://www.bbc.com/news/magazine-24459279} for some interesting anecdotes about this fact.}

\subsection{Using and Proving Statements With Nested Quantifiers}

To use and prove statements with nested quantifiers, you just apply the rules we have already introduced recursively.  To prove $\forall x \in \mathcal{U}: \exists y \in \mathcal{V}: P(x,y)$ we would:

\begin{fitch*}
	\textrm{Let $x_1 \in \mathcal{U}$ be arbitrary.}\\
	\textrm{Give a proof that $\exists y \in \mathcal{V}: P(x_1,y)$ is true.}\\
	\textrm{Conclude $\forall x \in \mathcal{U}: \exists y \in \mathcal{V}: P(x,y)$.}
\end{fitch*} 

However, what does a proof that $\exists y \in \mathcal{V}: P(x_1,y)$ look like?  Let's fill it in!

\begin{fitch*}
	\textrm{Let $x_1 \in \mathcal{U}$ be arbitrary.}\\
	\textrm{Construct a candidate $y_1 \in \mathcal{V}$ which might depend on $x_1$. }\\
	\textrm{Give a proof that $P(x_1,y_1)$ is true.}\\
	\textrm{Conclude $\forall x \in \mathcal{U}: \exists y \in \mathcal{V}: P(x,y)$.}
\end{fitch*} 

Note that since we are trying to find a $y_1 \in \mathcal{V}$ for which $P(x_1,y)$ is true when $y = y_1$, we might need to choose a different value of $y_1$ for each value of $x_1$.  For this reason we sometimes say that $y_1$ might depend on $x_1$. 



\begin{xca}	
	\begin{enumerate}
		\item It is true that $\forall n \in \mathbb{N}: 6 \divides (7^n-1)$.  Use this fact to argue that $6 \divides 342$.
		\item It is true that there is a nonzero function $f: \mathbb{R} \to \mathbb{R}$ which satisfies $ f''(x) = f'(x)+f(x)$ for all $x$.  Use this fact to argue that there are infinitely many functions different functions satisfying the same differential equation.
		\item Argue that $\forall n \in \mathbb{Z}:  4 \divides (12n)$.
		\item Argue that $\exists n \in \mathbb{Z}: (n-3) \divides n$.
		\item Say you know (somehow) that $x$ is an even number.  Argue that $3x$ is also even.
		\item Say you know (somehow) that both $12|x$ and $15|y$.  Argue that $3|(x+y)$.
		\item Argue that $\forall \epsilon \in (0,1) \exists \delta \in (0,1): 4(3+\delta)-4(3) < \epsilon$.
		\item Argue that $\exists x \in \mathbb{Z}: \forall y \in \mathbb{Z}:  y \divides x$.
		\end{enumerate}
	\end{xca}

\begin{solutions}	
	\begin{enumerate}
		\item Since we know $\forall n \in \mathbb{N}: 6 \divides (7^n-1)$ is true (we were told to assume this), we also know that $6 \divides (7^3-1)$ is true. Since $7^3-1 = 342$, we know that $6$ divides $342$ without even needing to check!  We could check that in fact $342 = 6 \cdot 57$, but this check is not necessary if we believe the theorem that$\forall n \in \mathbb{N}: 6  \divides  (7^n-1)$. 
		
		\item Since we know there is such a function, pick one and call it $f$.  Then for any constant $c \in \mathbb{R}$, you can check that the function $g(x) = cf(x)$ satisfies $g''(x) = g'(x)+g(x)$.  We don't need to know what $f$ is for this argument to be valid:  we just need to know that such an $f$ exists.
		
		\item Choose an integer arbitrarily and call it $n_1$.  We want to argue that $4 \divides (12n_1)$.  In other words, we need to find an integer $k_1$ for which $12n_1 = 4k_1$.  We can check that $k_1 = 3n_1$ works:  $12n_1 = 4(3n_1) = 4k_1$.

		\item We just need to find an $n_1$ which works.  Choosing $n_1 = 4$, we see that $n_1 - 3 = 1$, and it is clear that $1 \divides  4$.  We could have also verified this using $6$  to witness the proposition.  You might enjoy trying to find all of the solutions to this equation, and trying to justify that you have them all!
		
		\item Since $x$ is even we know that there is an integer $n$ with $x =2n$.  Use existential elimination to produce a particular $n_1 \in \mathbb{Z}$ with $x =2n_1$.  Then $3x  =3(2n_1) = 2(3n_1)$.  Since $3n_1$ is an integer, then $3x$ is even.  Notice that we are using existential introduction here:  we have argued that $3x$ is even (which means $\exists k \in \Z : x = 2k$) by constructing a candidate $k=3n_1$ and arguing that $x=2k$ is actually true for this choice.
		
		\item Since $12|x$ and $15|y$ we know that there exist integers $j$ and $k$ so that $x = 12j$ and $y=15k$.  Use existential elimination to produce particular $j_1 \in \Z$ and $k_1 \in \Z$ with $x = 12j_1$ and $y=15k_1$.  Then $x+y = 12j_1+15k_1 = 3(4j_1+5k_1)$.  Since $(4j_1+5k_1)$ is an integer, we have shown that $3|(x+y)$.  Notice that we are using existential introduction here:  we have argued that $3|(x+y)$ (which means $\exists t \in \Z: x+y = 3t$) by constructing a candidate $t = 4j_1+5k_1$ and showing that $x+y = 3t$ is actually true for this choice.
		 
		\item Let $\epsilon_1 \in (0,1)$ be arbitrary.  We are trying to construct a $\delta \in (0,1)$ for which $4(3+\delta)-4(3) < \epsilon$.  This inequality is equivalent to the inequality $4\delta < \epsilon$.  So we just need to choose a $\delta$ which is less than $\epsilon_1/4$.  To be explicit about it, lets choose $\delta_1 = \epsilon_1/8$.  Then 
		\begin{align*}
			4(3+\delta_1)-4(3) &= 4\delta_1 \\
			&= 4(\epsilon_1/8)\\
			&=\epsilon_1/2\\
			&<\epsilon
		\end{align*}
		
		Note the logic here:  for an arbitrarily chosen $\epsilon$, we were able to cook up a $\delta$ (dependent on $\epsilon$) which satisfied the inequality.
		\item We need a witness for this existentially quantified statement.  I will choose $x=0$.  So we are trying to argue that $\forall y \in \mathbb{Z}:  y \divides 0$.  Let $y_1$ be an arbitrary integer.  We want to show that $y_1 \divides 0$.  So we need to find an integer $k$ for which $0 = y_1 \cdot k$.  $k=0$ works!  Thus there is an integer (namely zero) which is divisible by all other integers.
	\end{enumerate}
\end{solutions}





\chapter{Logical Connectives}

\section{Introduction}

In the last chapter we learned three tools for building new predicates and propositions out of old propositions:  substitution of a variable with another variable or value, universal quantification of a variable, and existential quantification of a variable.

We can also create new predicates from old ones by using the logical connectives ``and'' ($\wedge$),``implies'' ($\implies$), ``if and only if'' ($\iff$), ``or'' ($\vee$), and ``not'' ($\neg$).

For instance if $P$ is the proposition ``It is raining'' and $Q$ is the proposition ``I have no umbrella'', then we can form the compound sentence $P \wedge Q$ which says ``It is raining and I have no umbrella''.

These logical connectives are used in our everyday language, but we do not use them with the precision which is required for mathematics or computer science.  Consider the following sentence:

\begin{quote}
		``I will go out for pizza or I will go out for ice cream''
\end{quote}

This sentence is ambiguous.  It is unclear whether the person saying this sentence is allowing for the possibility that they will get both pizza and ice cream, or if they are claiming that they will only get one and not the other.

The other connectives suffer from similar ambiguities in natural language.  We will define each of these connectives precisely by describing how they interact with the truth values of the statements which they connect.

To see the importance of resolving such ambiguities consider the following theorem:

\begin{theorem}[Euclid's Lemma]
		Let $p$ be a prime number.  Let $a$ and $b$ be integers. If $p$ divides $ab$, then $p$ divides $a$ or $p$ divides $b$
\end{theorem}

In order to understand what this theorem is saying we need to understand what ``If ...., then...'' means precisely.  Is the theorem making any claim about what happens if $p$ does not divide $ab$?  We also need to understand what the ``or'' means precisely.  Is the theorem claiming that $p$ must divide only one of $a$ or $b$, but not both?  Or does it leave open the possibility that $p$ divides both of them?

You can see that having a precise and consistent meaning for these logical connectives is important for communicating mathematical ideas. 

Another major goal of this chapter is to integrate an understanding of how to both \textbf{use} (eliminate) and \textbf{prove} (introduce) statements involving each of the connectives.  We will learn the elimination and introduction rules associated with each connective.  These will show us how to start building natural deduction proof outlines for statements involving these connectives.

In writing computer programs we run into similar issues.  We often want the computer to execute some command depending on whether a set of different conditions are satisfied.  Consider the simple problem of instructing a computer to sum all of the numbers between $1$ and $100$ which are divisible by $2$ or $3$ but not both.

In psuedocode, we might write

\begin{lstlisting}[language=Python]
set SUM = 0
for k in {1, 2, 3, ..., 100}:
 If ((k%2==0) OR (k%3==0)) AND (NOT((k%2==0) AND (k%3==0))):
  SUM := SUM + k 
return SUM
\end{lstlisting}

We cannot write functional code like this without understanding the precise behaviour of the connectives OR, AND, and NOT.



\section{Conjunction}

Given two statements $L$ and $R$, we can form a new  statement called the \index{Conjunction}\textbf{conjunction} of $L$ and $R$.  We will write $L \wedge R$ for this new statement.  We read $L \wedge R$ as ``$L$ and $R$''.  We will also sometimes refer to $L$ as the \index{conjunct} \textbf{left conjunct} and $R$ as the \textbf{right conjunct}.

We will define conjunction by explicitly showing how the truth value of the conjunction can be obtained from the truth value of each conjunct.

\begin{definition}
		Let $L$ and $R$ be two statements.  We define  $L \wedge R$ as a sentence whose truth value is determined by the following table:
		
		
		\begin{table}[h!]
			\begin{center}
				\caption{Truth Table for Conjunction}
				\begin{tabular}{c|c|c} 
					$L$ & $R$ & $L \wedge R$ \\
					\hline
					$\F$ & $\F$ & $\F$ \\ 
					$\F$ & $\T$ & $\F$ \\ 
					$\T$ & $\F$ & $\F$ \\ 
					$\T$ & $\T$ & $\T$ \\ 
				\end{tabular}
			\end{center}
		\end{table}
	
	\end{definition}

This truth table makes it explicitly clear that a conjunction is only true when both of its conjuncts are true, and is false otherwise.

\subsection{Using a conjunction}  If we know that the conjunction $L \wedge R$ is true, then we know (from looking at the truth table) that both $L$ and $R$ are each true individually.  So if we know that a conjunction is true, we can use that each of its conjuncts are also true.

In a proof, if we know that $L \wedge R$ is true, then we may cite that fact that $L$ is true or that $R$ is true whenever we want in our argument.  This is called \textbf{conjunction elimination} because it takes a hypothesis which includes a $\wedge$ and ``eliminates'' it to obtain a new hypothesis without the $\wedge$.

\subsection{Proving a conjunction}  To show that a conjunction is true, we need to demonstrate that both $L$ is true and $R$ is true.  We will do this in our proof outline as follows:

To prove $p \wedge q$:

\begin{fitch*}
	\textrm{(Need to show $L \wedge R$ is true)}\\
	\textrm{Put a proof of $L$ here.}\\
	\textrm{Put a proof of $R$ here.}\\
	\textrm{Conclude that $L \wedge R$ is true.}\\
\end{fitch*}

This is also called ``conjunction introduction'', because it allows us to introduce $L \wedge R$ as a known statement in our arguments.

\begin{xca}
		If the sentence is a proposition, evaluate its truth value with justify your answer.  If the sentence is a predicate identify its free variables, try to find some values for the free variables which make it true, and some which make it false (if possible).
		
		\begin{enumerate}
				\item $(6 < 7 ) \wedge (6\divides 18)$
				\item $(2+2=4) \wedge (9 \divides 3)$
				\item $(6 \textrm{ is even }) \wedge (7 \textrm{ is odd})$
				\item $[(1+1=2) \wedge (2+2 = 4)] \wedge (4+4 =8)$
				\item $(x \divides y) \wedge (y \divides x)$
				\item $(x \textrm{ is even }) \wedge (y \textrm{ is odd})$ 
				\item $\forall x: [(x \cdot 0 = 0) \wedge (x \cdot 1 = x)]$
				\item $\exists x: [(2x+1 = 5) \wedge (3x+3 = 5)]$
				\item $[\exists x: (2x+1 = 5)] \wedge [\exists x: (3x+3 = 5)]$
			\end{enumerate}
	\end{xca}

\begin{solutions}
	
	\begin{enumerate}
		\item $(6 < 7 ) \wedge (6 \divides 18)$ - True proposition. Both conjuncts are true, so the conjunction is true.  Note the ``recursive'' nature of the evaluation of the truth value here:  to be explicit, I would need to verify that $6\divides 18$ is true by demonstrating the witness $k= 3$ for the existentially quantified statement $\exists k : 18 = 6k$.
		\item $(2+2=4) \wedge (9\divides3)$ - False proposition.  The right conjunct is false.
		\item $(6 \textrm{ is even }) \wedge (7 \textrm{ is odd})$ - True proposition.  Both conjuncts are true.
		\item $[(1+1=2) \wedge (2+2 = 4)] \wedge (4+4 =8)$ - True proposition.  We need to evaluate this in stages:
		
		\begin{align*}
		[(1+1=2) \wedge (2+2 = 4)] \wedge (4+4 =8) &= [\T \wedge \T] \wedge \T\\
		&= \T \wedge \T\\
		& = \T
		\end{align*}
		
		\item $(x\divides y) \wedge (y \divides x)$ - This is a predicate with free variables $x$ and $y$.  It is true if we set $(x,y) = (4,-4)$.  It is false if we set $(x,y) = (2,4)$, since in that case the right conjunct is false.
		\item $(x \textrm{ is even }) \wedge (y \textrm{ is odd})$   - This is a predicate with free variables $x$ and $y$. It is true if we set $(x,y) = (2,7)$.  It is false if we set $(x,y) = (3,3)$, since in that case the left conjunct is false.
		\item $\forall x: [(x \cdot 0 = 0) \wedge (x \cdot 1 = x)]$ - This is a true proposition. 
		
		\begin{fitch*}
			\textrm{Let $x_1  \in \mathbb{R}$ be arbitrary.}\\
			\textrm{$x_1 \cdot 0 = 0$ is true, since any number multiplied by $0$ is $0$. }\\
			\textrm{$x_1 \cdot 1 = x_1$ is true, since any the product of any number with $1$ is  itself.}\\
			\textrm{Since both conjuncts are true, the conjunction $(x_1 \cdot 0 = 0) \wedge (x_1 \cdot 1 = x_1)$ is true}\\
			\textrm{Since $x_1$ was chosen arbitrarily, the universally quantified statement}\\
			 \textrm{ \hphantom{dsada} $\forall x \in \mathbb{R} [(x \cdot 0 = 0) \wedge (x \cdot 1 = x) ]$ is true.} 
			\end{fitch*}
		\item $\exists x \in \mathbb{R}: [(2x+1 = 5) \wedge (3x+3 = 5)]$ - This is a false proposition.  There is no number $x$ which will make $2x+1 = 5$ and $3x+3 = 5$ both true statements.  The reason is that if $2x+1  = 5$, then $x$ must be $2$.   However if $3x+3 = 5$, then $x$ must be $\frac{2}{3}$.  There is no number which is both equal to $2$ and $\frac{2}{3}$, since $2 \neq \frac{2}{3}$.
		\item $[\exists x \in \mathbb{R}: (2x+1 = 5)] \wedge [\exists x \in \mathbb{R}: (3x+3 = 5)]$ - This is a true proposition. $[\exists x: (2x+1 = 5)] $ is true since $x=2$ witnesses the truth of it.   $[\exists x: (3x+3 = 5)]$ is true since $x= \frac{2}{3}$ witness the truth of it. Since both conjuncts are true, the conjunction is true.
	\end{enumerate}



\end{solutions}


\section{Implication}

Implication is one of the hardest logical connectives to internalize the meaning of.  Most people find it to be extremely unintuitive. However, it is also one of the most important logical connectives.

\subsection{Intuition for implication}

The contents of this subsection are intuitive rather than rigorous.  We need some idea of what we are trying to accomplish with the implication connective before we give a rigorous definition.  So understand everything in this subsection with that caveat in mind.

Let $H$ and $C$ be two statements. Let us call $H$ the ``hypothesis'' and $C$ the ``conclusion''.

We want to define $H \implies C$ (read: ``H implies C'') as the sentence


	``There is a valid argument allows us to conclude $C$ from a hypothesis of $H$.''


 If we can produce such an argument, the implication should be true.  If it is impossible produce such an argument, the implication should be false.

Lets consider some examples.

Should $(1+1 = 2) \implies 2+2 = 4$ be true or false?  If we assume that $1+1 = 2$, then we could multiply both sides by $2$ to obtain $2+2 = 4$.  So it seems that we have a valid argument that if $1+1 = 2$, then $2+2 = 4$.  So the implication $(1+1 = 2) \implies 2+2 = 4$ should be true.

Should $(1+1 = 0) \implies (1+1 = 1)$ be true or false?  This is less straightforward.  Notice that we are not making any claim that either the hypothesis or conclusion is true.  We only want to see if we could make a valid argument which starts from the hypothesis and leads to the conclusion.

In this case, I think I can make an argument:

\begin{align*}
&\textrm{If } 	1+1 =0\\
&\textrm{then } 	2(1) = 0\\
&\textrm{so }	1 =0\\ 
\end{align*}

but then

\begin{align*}
	1+ 1 & = 0+ 0 \textrm{ since $1=0$}\\
	&= 0 \textrm{ since $0+0 =0$}\\
	&= 1 \textrm{ since $0 = 1$}
\end{align*}

So the claim that ``$(1+1 = 0)$ implies $(1+1 = 1)$'' should be true.

Should $(1+1 = 0) \implies (0 = 0)$ be true or false? I think we can make a valid argument:

\begin{align*}
	&\textrm{If } 1+1 = 0 \\
	&\textrm{then } 0(1+1)  = 0(0) \textrm{ since we can multiply both sides by $0$}\\
	&\textrm{so } 0=0 
\end{align*}

Thus, $(1+1 = 0) \implies (0 = 0)$ is true.

Should $(1+1 = 2) \implies (1 = 0)$ be true or false?  This should be false.  No matter how clever I am, there will be no valid argument which starts from a true hypothesis and leads to a false conclusion.

To steal from philosopher Bertrand Russel \footnote{\url{http://ceadserv1.nku.edu/longa//classes/mat385_resources/docs/russellpope.html}}, should  $$(1+1 = 1) \implies \textrm{``Bertrand Russel is the pope''}$$ be true or false?  This is more difficult.  It is hard to say whether we could come up with an argument which starts from the hypothesis that $1+1 = 1$ and arrives at the conclusion that Bertrand Russel is the pope, but it is also difficult to say that such an argument is impossible to make.  Here is Bertrand's solution:

\begin{quote}
		Assume that $1+1 = 1$.
		
		The pope is one person, and I am another.
		
		Since $1+1 = 1$, then I and the pope together are one.
		
		Thus I am the pope.
	\end{quote}

This was a very tricky argument.  It might be even harder to decide whether we could come up with a valid argument in other circumstances.  For instance, can you come up with an argument that if $E = mc^3$ then cows are made of diamonds?  I wouldn't know how to make such an argument, but I also wouldn't rule out such an argument existing.  This would put us in the awkward situation of being unable to evaluate the truth value of this statement unless someone comes around who is clever enough to make the argument.  It would also leave open the possibility that this argument is impossible to make, but for reasons of content, rather than the logical form of the statements.

The only thing we can really be sure of with implications (used in this intuitive sense) is that if a hypothesis is false, then a valid argument can sometimes lead us to true conclusions, and can sometimes lead us to false conclusions.  However, when we start with a true premise then a valid argument will only ever lead to true conclusions, not false ones.

\subsection{Rigorous definition of implication}

The intuitive idea of implication which we covered in the last subsection is appealing, and conforms well to our standard English usage of the words ``If ...., then ....''.

However, as we saw, our intuitive view of implication is not well equipped to handle implications where the hypothesis is false.

We will sweep these issues  under the rug by providing a definition of implication which is less intuitive but is easier to work with.  Namely, we will define implication by explicitly showing how the truth value of the implication can be obtained from the truth value of the hypothesis and conclusion.

\begin{definition}
	Let $H$ and $C$ be two statements.  We define the new  $H \implies C$ as a sentence whose truth value is determined by the following table:
	
	
	\begin{table}[h!]
		\begin{center}
			\caption{Truth Table for Conjunction}
			\begin{tabular}{c|c|c} 
				$H$ & $C$ & $H \implies  C$ \\
				\hline
				$\F$ & $\F$ & $\T$ \\ 
				$\F$ & $\T$ & $\T$ \\ 
				$\T$ & $\F$ & $\F$ \\ 
				$\T$ & $\T$ & $\T$ \\ 
			\end{tabular}
		\end{center}
	\end{table}
\end{definition}

Notice that with this convention, we can always make a valid argument to establish any conclusion if we start with a false hypothesis.  These implications (recorded in the first two rows of the truth table) are called \index{vacuous truth}\textbf{vacuously true}.  These two rows of the truth table are extremely unintuitive for most people.  You will need to do serious internal work to integrate these ideas into your being.

However, if the implication is true, then a true hypothesis can only ever establish a true conclusion.

If you think that you have derived a false conclusion from a true hypothesis, then you are wrong.  Your argument is not valid, and the implication must be false.

\subsection{Some further justification for this strange definition}

Another motivation for defining the truth table for implication in this way stems from how nicely it interacts with quantifiers.

We want to be able to say that if a real number is even, then that number plus one is odd:

\[
 \forall x \in \mathbb{Z}: \textrm{$x$ is even} \implies \textrm{$x+1$ is odd}
\]

If we want universal elimination to work nicely with implication, then we should get a true statement when we substitute any integer in for $x$.  Substituting $x = 7$ gives the statement:

\[
\textrm{$7$ is even} \implies \textrm{$7+1$ is odd}
\]

Both the hypothesis and conclusion are false, but we want the implication to be true!  This justifies the first row in the truth table.  It also agrees with our intuitive perspective.  If $7$ \textit{were} even, then since an even number plus $1$ is odd, we would have that $7+1 = 8$ is odd.  The argument is valid even though both the hypothesis and conclusion are false.

Similarly we want to be able to say that if a real number is less than $6$, then it is less than $8$.

\[
\forall x \in \mathbb{R}: (x< 6) \implies (x<8)
\]

If this statement is true, then it should be true when we substitute any real value for $x$ that we wish.  Substituting $x  = 7$ gives

\[
(7< 6) \implies (7 < 8)
\]

Here the hypothesis is false, but conclusion is true, and we want the implication to be true!  This justifies the second row in the truth table.  It also makes sense from our intuitive perspective:  if $7$ \textit{were} less than $6$, surely it would also be less than $8$.  The argument is valid, even though the hypothesis is false.  We reached a correct conclusion ``by accident''.

In short, if we want universal elimination to work with implication then we need to have the first two rows of the truth table for implication be the way they are.


\begin{xca}
	Which of the following implications are true?  If the implication is true, try to find an ``argument'' which starts with the hypothesis and leads to the conclusion.  If the implication is false, explain why such an ``argument'' must be impossible to find.
		\begin{enumerate}
			\item If $1<2$ then $6<5$.
			\item If $2<1$ then $6<5$.
			\item If $2 = 1$ then $0 = 0$.
			\item If $2 = 2$ then $4 = 4$.
			\item If $1$ is odd then $2$ is even.
			\item If $1$ is even then $2$ is odd.
			\item If $1$ is odd then $2$ is odd.
			\item If $1$ is even then $2$ is even. 
		\end{enumerate}
\end{xca}

\begin{solutions}
	\begin{enumerate}
		\item If $1<2$ then $6<5$. - The hypothesis is true while the conclusion is false.  Thus there cannot be a valid argument starting with the hypothesis and ending with the conclusion.  Hence this implication is false.
		\item If $2<1$ then $6<5$.  - Since the hypothesis is false, this implication is vacuously true.  While it is not logically necessary to make an argument (this is really part of our definition of valid argument), it is psychologically comforting to do so.  We can start with the premise that $2<1$, add $4$ to both sides, and conclude that $6<5$.
		\item If $2 = 1$ then $0 = 0$. - Since the hypothesis is false, this implication is vacuously true. If we assume $2 = 1$, then multiplying both sides by $0$ yields the statement $0=0$.
		\item If $2 = 2$ then $4 = 4$. - Both hypothesis and conclusion are true, so the implication is true.  A valid argument is to multiply both sides of $2=2$ by $2$.
		\item If $1$ is odd then $2$ is even.  - Both hypothesis and conclusion are true, so the implication is true.  A valid argument is to say that since one more than an odd number is even, and $1$ is odd, then $2$ is even.
		\item If $1$ is even then $2$ is odd. - The hypothesis is false, so this implication is vacuously true. A valid argument is to say that since one more than an odd number is even, if $1$ is even, then $2$ is odd.
		\item If $1$ is odd then $2$ is odd. - The hypothesis is true and the conclusion is false, so this implication is false.  There is no valid argument which can conclude false statements from true premises.
		\item If $1$ is even then $2$ is even.  - The hypothesis is false, so this statement is vacuously true.  A valid argument might not even cite that $1$ is even.  Assume $1$ is even, and then just argue that since $2 = 2(1)$, then by definition $2$ is even.
	\end{enumerate}
\end{solutions}

\begin{xca}
Evaluate the truth values of the following expressions.
		\begin{enumerate}
			\item $(\F \implies \F) \implies \T$
			\item $\T \implies (\T \implies \F)$
			\item $\T \implies (\T \implies \T)$
			\item $(\T \implies \F) \implies (\F \implies \T)$
			\item $\F \implies (\F \implies \F)$
			\item $\T \implies (\T \implies \F)$
		\end{enumerate}	
\end{xca}

\begin{solutions}
	Evaluate the truth values of the following expressions.

		\begin{enumerate}
	\item \begin{align*}(\F \implies \F) &\implies \T \\  \T &\implies \T \\ &  \hphantom{ds}\T\end{align*}
	\item \begin{align*}\T &\implies (\T \implies \F) \\ \T &\implies \F \\  & \hphantom{ds}\F \end{align*}
	\item \begin{align*}\T &\implies (\T \implies \T) \\  \T &\implies \T  \\  &\hphantom{ds}\T\end{align*}
	\item \begin{align*}(\T \implies \F) &\implies (\F \implies \T) \\ \F &\implies \T \\ &\hphantom{ds}\T\end{align*}
	\item \begin{align*}\F &\implies (\F \implies \F) \\ \F &\implies \T \\ &\hphantom{ds}\T\end{align*}
	\item \begin{align*}\T &\implies (\T \implies \F) \\ \T &\implies \F \\ &\hphantom{ds}\F\end{align*}
\end{enumerate}	

\end{solutions}


\subsection{Using an implication}

Knowing that the implication $H \implies C$ is true does not tell us whether $H$ is true or $C$ is true.  It only tells us that \textbf{if} $H$ is true, \textbf{then} $C$ must be true as well.

If we know both that $H$ is true and $H \implies C$ is true, then we can conclude that $C$ is true.  This kind of reasoning is also called \textbf{implication elimination} or \index{``modus ponens''}\textbf{modus ponens} (from ``modus ponendo ponens'' which is Latin for "mode that by affirming affirms"). There is only one row of the truth table where both $H$ and $H \implies C$ are both true, and in that row $C$ is also true.


What happens if we know an implication $H \implies C$ is true, and we know that $H$ is false?  This is a sad situation to be in, because the implication is useless for making arguments.  We can see that in this case, $H$ could be true or false, so we gain no information about $C$ in this case.  Similarly, knowing that $C$ is true gives us no information about $H$.

\begin{example}
		Here is a theorem which should be familiar to you form high school:
		
		\begin{theorem}[Linear Factors Theorem]
			Let $p(x)$ be a polynomial with real coefficients and let $a$ be a real number.  If $p(a) = 0$, then there exists another polynmial $g(x)$ with real coefficients such that $p(x) = (x-a)g(x)$. 
		\end{theorem}
		
		Maybe I am interested in the behaviour of the rational function $f(x) = \frac{x^7+x-2}{x-1}$ near $x=1$ as either a standalone problem in a Calculus course, or as a small part of some real mathematical work I am doing.
		
		Let $p(x) = x^7+x-2$ The linear factors theorem says that the implication $(p(1) = 0) \implies \exists g: [p(x) = (x-1)g(x)]$ is true.  $p(1) = 0$ is actually true because $p(1) = 1^7+1-2 = 0$.  So, by Modus Ponens we know that there is a polynomial $g$ with $p(x) = (x-1)g(x)$.
		
		The theorem doesn't actually tell me how to find $g$ (there is an algorithm called ``polynomial long division'' which lets you find $g$), but that might not be necessary for my problem.  The very fact that $g$ exists allows me to conclude that
		
		$$f(x) = \frac{(x-1)g(x)}{x-1} = g(x)$$
		
		when $x \neq 1$, and so we can tell that $f$ has a removable discontinuity at $x=1$.
		
	\end{example}

\subsection{Proving an implication}

If we want to convince someone that an implication $H \implies C$ is true, what do we need to do?

If the hypothesis $H$ is false, then the implication is automatically true (no matter whether the conclusion $q$ is true or false).  Remember that we call this kind of implication ``vacuously true''.  

So to convince someone that an implication $H \implies C$ is true, we do not need to ever consider the case when $H$ is false.  We only have to consider what happens when $H$ is true.  If $H$ is true, then for the implication to be true we need to show that $C$ is also true.

In other words, to prove an implication $H \implies C$ we should \textbf{assume} (or pretend) that $H$ is true, and try to argue that $C$ must be true relative to that assumption.

In our proof outline, we will record the fact that we have made an \textbf{assumption} by initiating a vertical bar with an indent.  Every part of the argument within the scope of that vertical bar is made relative to the assumption (so we can always pretend that the assumption is true for those parts of the argument).  We cannot assume that the assumption is true in other places of our argument!  When we have finished proving the implication, we end the vertical bar, and unindent.

So our proof outline looks like this:

\begin{fitch*}
	\textrm{(Need to show $H \implies C$ is true)}\\
	\textrm{Assume $H$ is true.}\\
	\fa \textrm{ Argue that $C$ is true, assuming that $H$ is true}\\
	\fa \textrm{ Continue arguing that $C$ is true, still under the assumption that $H$ is true.}\\
	\fa \textrm{ Conclude that $C$ is true.}\\
	\textrm{Conclude that $H \implies C$ is true.}\\
\end{fitch*}

Note:  you \textbf{cannot} use that either $H$ or $C$ are true.  We only proved that if $H$ is true, then $C$ is.  We didn't actually argue that either one was true.

\begin{example}
		Let's prove that if $n$ is an odd integer, then $n+7$ is an even integer.
		
		Symbolically, we are saying
		
		\[
		 \forall n \in \mathbb{Z}: (\textrm{$n$ is odd}) \implies (\textrm{$n+7$ is even})
		\]
		
		\begin{fitch}
				\textrm{Let $n_1$ be an arbitrary integer.}\\
				\textrm{Assume $n_1$ is odd.}\\
				\fa \textrm{There is an integer $k$ so that $n_1  = 2k+1$.  Choose one such $k$ and call it $k_1$. }\\
				\fa \textrm{Then $n_1+7 = (2k_1+1)+7$}\\
				\fa \textrm{So $n_1+ 7 = 2k_1 +8$}\\
				\fa \textrm{So $n_1+ 7 = 2(k_1+4)$}\\
				\fa \textrm{So $n_1 + 7$ is even.}
			\end{fitch}
		
		Commentary:
		
		\begin{enumerate}
			\item We are introducing an arbitrarily chosen integer to introduce the universal quantifier.  If we can argue that the theorem is true for this  ``totally random'' integer $n_1$, then we can be sure it is true of all integers.
			\item If $n_1$ is not odd, then the theorem is vacuously true.  So we only need to consider the case that $n_1$ is odd, and try to prove that $n_1+7$ is even in that case.  This is implication introduction.
			\item Here we are using the definition of even.  Since the definition of even is an existentially quantified statement, when we use it (eliminate the existential quantifier) we obtain a witness $k_1$ which we know nothing about except for the fact that it is a witness.
			\item Algebra
			\item Algebra
			\item Algebra
			\item Since we have demonstrated that $n_1+7 = 2(k_1+4)$, we have shown that $m_1 = k_1+4$ is a witness for the existentially quantified statement $\exists m : (n_1+7) = 2m $.  This is a proof that $n_1+7$ is even.
			\end{enumerate}
	\end{example}

\subsection{The Principle of Explosion}

One consequence of our definition of implication is that if you derive an absurdity (if you give an argument for $F$), then you can derive any other statement.  

Since $\F \implies P$ for any proposition $P$, then if we derive $\F$ somehow, we can conclude via Modus Ponens that $P$ must be true.

This formalizes Bertrand Russel's intuition that if $1+1 = 2$, then he is the pope.

This is actually a common sort of expression in our everyday use of language as well.  You have probably heard someone say something like ``Oh ya?  If that is a real Lamborghini, then I'm the king of England''.  This expression indicates that if one thing is false (this case is a Lamborghini) then we could argue any other statement from that, no matter how absurd.

This principle is kind of a technicality, but it is a useful one.  We will see that it allows us to have uniform arguments for case analysis, even when one of the cases leads to a contradiction.

\section{Biconditional}

Consider the following three sentences:

\begin{enumerate}

\item ``If you do your homework, then you will get a good grade in the course.''

\item  ``You can only get a good grade in the course if you do your homework.''

\item ``You will get a good grade in the course if and only if you do your homework.''

\end{enumerate}

Let $H(x)$ be the predicate ``$x$ does their homework'' and $G(x)$ be the predicate ``$x$ will get a good grade in the course''.

Then these sentences correspond to the following symbolic propositions:

\begin{enumerate}
		\item $\forall x: H(x) \implies G(x)$
		\item $\forall x: G(x) \implies H(x)$
		\item $\forall x: [(H(x) \implies G(x)) \wedge (G(x) \implies H(x))]$
	\end{enumerate}

\begin{xca}
		Kira is a student in the class.  For each of the following statements, determine whether they are consistent with (1), (2), or (3) being true.  Explain.
		
		\begin{enumerate}
			\renewcommand{\theenumi}{\alph{enumi}}
			\item ``Kira did their homework and got a good grade.''
			\item ``Kira did their homework and didn't get a good grade.''
			\item ``Kira didn't do their homework and got a good grade.''
			\item ``Kira didn't do their homework and they didn't get a good grade.''
			\end{enumerate}
	\end{xca}

\begin{solutions}
			\begin{enumerate}
		\renewcommand{\theenumi}{\alph{enumi}}
		\item ``Kira did their homework and got a good grade.'' - This is consistent with all three sentences.
		\item ``Kira did their homework and didn't get a good grade.'' - This is consistent with sentence (2) since (2) only tells you what happens if you do get a good grade.  It doesn't say anything about what happens if you do not get a good grade.  It is inconsistent with both (1) and (3), which both claim that if someone does their homework then they must get a good grade.
		\item ``Kira didn't do their homework and got a good grade.'' - This is consistent with sentence (1) since (1) only tells you what happens if you do your homework.  It doesn't say anything about what happens if you do not do your homework. It is inconsistent with both (2) and (3), which both claim that if someone gets a good grade, then they must have done their homework.
		\item ``Kira didn't do their homework and they didn't get a good grade.''  - This is consistent with all three propositions.
	\end{enumerate}
	\end{solutions}

The third sentence is an example of a \textbf{biconditional} statement.  It makes two conditional claims  (implications) at the same time. 

\begin{definition}
	 	Let $P$ and $Q$ be two statements.  We define the biconditional of $P$ and $Q$ by the following formula:
	 	
	 	\[
	 	P \bi Q = (P \implies Q) \wedge (Q \implies P)
	 	\]
	\end{definition}

\begin{xca}
	Fill out the following table with $\T$ or $\F$.


	\begin{table}[h!]
	\begin{center}
		\caption{Truth Table for Biconditional}
		\begin{tabular}{c|c|c|c|c|c} 
			$P$ & $Q$ & $P \implies Q$ & $Q \implies P$ & $(P \implies Q) \wedge (Q \implies P)$ & $P \bi Q$ \\
						\hline
			$\F$ & $\F$ &  & &  & \\
						\hline
			$\F$ & $\T$ &  & &  & \\
						\hline
			$\T$ & $\F$ &  & &  & \\
						\hline
			$\T$ & $\T$ &  & &  & \\
		\end{tabular}
	\end{center}
\end{table}
\end{xca}

\subsection{Using a biconditional} 

If we know (somehow) that the biconditional $P \bi Q$ is true, then by definition we also know that $(P \implies Q) \wedge (Q \implies P)$ is true.  So we can use that $P \implies Q$ is true, and that $Q \implies P$ is true.  

\begin{example}
		The Pythagorean theorem is one of the most famous theorems in the world:
		
		\begin{theorem}[Pythagorean Theorem]
			A triangle with side lengths $a$, $b$, $c$ satisfies $a^2+b^2 = c^2$ if and only if one of the interior angles of the triangle is a right angle.
			\end{theorem}
		
		Since we know that this biconditional statement is true for any triangle, we can use both the forwards and backwards implications freely in our reasoning.  In forwards direction we can say that if a triangle has side lengths $5$, $12$, and $13$, then since $5^2 + 12^2 = 25+144 = 169$ and $13^2 = 169$, then we can be sure that this triangle is a right triangle.  This is useful if you want to make a right angle but do not have a square tool available:  take a long nonstretchy rope of length $5+12+13 = 30$ feet.  Tie it in a loop.  Mark off $5'$, $12'$, and $13'$ distances around the loop.  When you pull this tight at the markings to make a triangle, you can be sure that the angle opposite the longest side is a right angle.
		
		In the backwards direction, if you are cutting a $2''$ by $4''$ piece of lumber along a diagonal, and you need the diagonal to be $5''$ long, then the you know that you will need to cut $3''$ off of one side.  The reason is that if we let $x$ be the number of inches to be cut, then the backwards implication of the theorem tells us that $x^2+4^2  =5^2$.  We can solve this equation to see that $x=3$.
	\end{example}

\subsection{Proving a biconditional}

Since we have defined $P \bi Q$ as  $(P \implies Q) \wedge (Q \implies P)$, the proof outline for biconditional statements is just to prove $P \implies Q$ and then prove $Q \implies P$.

\begin{fitch}
	\textrm{Assume $P$}\\
	\fa \textrm{Prove $Q$}\\
	\textrm{Assume $Q$}\\
	\fa \textrm{Prove $P$}
	\end{fitch}


Note:  Sometimes mathematicians will refer to lines 1 and 2 as the ``forward" or ``only if" part of the argument, and lines 3 and 4 as the ``backward'' or ``if'' part of the argument.  These names make sense because the implication arrows are either pointing forward from $P$ to $Q$, or backwards from $Q$ to $P$.  If we write ``$P$ if and only if $Q$'', then ``$P$ only if $Q$'' represents $P \implies Q$ and ``$P$ if $Q$'' represents $Q \implies P$.

It is a very common mistake for students to forget the backward part of the argument when proving a biconditional. 

\begin{example}
		Lets prove that $0$ divides an integer if and only if that integer is $0$.
		
		Symbolically we are trying to show
		
		\[
		\forall n \in \mathbb{Z}: [ (0 \divides n) \bi (n=0)]
		\]
		
		\begin{fitch}
				\textrm{Choose an arbitrary integer and call it $n_1$}\\
				\textrm{ Assume $(0 \divides n)$}\\
				\fa \textrm{Then $n_1 = 0k$ for at least one integer $k$.  Call one such integer $k_1$.}\\
				\fa \textrm{Then $n_1= 0k_1$.}\\
				\fa \textrm{So $n_1=0$}\\
				\textrm{Assume $n_1=0$}\\
				\fa \textrm{Then $n_1 = 0 \cdot 1$}\\
				\fa \textrm{So $0 \divides n_1$}
		\end{fitch}
	
	Commentary:
	
	\begin{enumerate}
			\item Universal introduction.
			\item This line starts the proof of the ``forwards half''.
			\item Definition of divisibility.
			\item We used existential elimination to produce the witness $k_1$.
			\item We used algebra to conclude $n_1 = 0$.  This is the conclusion we want to reach, so the ``forwards half'' of the argument is now complete.
			\item This line starts the proof of the ``backwards half''.
			\item This is a true fact.
			\item We are using $1$ as a witness to introduce the existential quantifier $\exists k: n_1 = 0 \cdot k$, which is the definition of $0 \divides n_1$.
		\end{enumerate}
	\end{example}



\section{Disjunction}

We remarked in the introduction that our ordinary language use of the work ``or'' is ambiguous.  

Consider the following sentence:

\begin{quote}
	``I will go out for pizza or I will go out for ice cream''
\end{quote}

This sentence is ambiguous.  It is unclear whether the person saying this sentence is allowing for the possibility that they will get both pizza and ice cream, or if they are claiming that they will only get one and not the other.

In mathematics and computer science whenever we use the word ``or'' without clarifying we always mean the \index{Inclusive Or}\textbf{``inclusive or''}, which permits both constituent statements to be true.  There is a notion of \index{Exclusive Or}\textbf{``exclusive or''} which is false when both constituent statements are true.  This exclusive version is also called XOR.  We will not have further need of XOR in this text since XOR can be defined in terms of other logical connectives.

\begin{definition}
		Let $L$ and $R$ be two predicates or quantifiers.  We define the \index{disjunction}\textbf{disjunction} of $L$ with $R$ by the following truth table.  We read the symbol $L \vee R$ as ``$L$ or $R$''.  We call $L$ the ``left disjunct'' and $R$ the ``right disjunct''.
		
				\begin{table}[h!]
			\begin{center}
				\caption{Truth Table for Disjunction}
				\begin{tabular}{c|c|c} 
					$L$ & $R$ & $L \vee R$ \\
					\hline
					$\F$ & $\F$ & $\F$ \\ 
					$\F$ & $\T$ & $\T$ \\ 
					$\T$ & $\F$ & $\T$ \\ 
					$\T$ & $\T$ & $\T$ \\ 
				\end{tabular}
			\end{center}
		\end{table}
	\end{definition}

\subsection{Using a Disjunction}

If I want to argue that $C$ is true, and I know $L \vee R$ is true, how should we proceed?  An example might help:

\begin{xca}
		Imagine you are playing a game with your friend.  They have a health score of $3/20$.  You have two cards in your hand, one of which deals $5$ points of damage and the other deals $7$ points of damage.  You know you will be playing at least one card this turn.  You want to convince your friend that they are about to lose the game.  What argument do you make?
\end{xca}

\begin{solutions}
		To convince your friend that they are about to lose, you should look at both cases.  \textbf{If} you play the first card, \textbf{then} they will sustain $5$ points of damage and lose the game.  \textbf{If} you play the second card, \textbf{then} they will sustain $7$ points of damage and lose the game.  Since they will lose the game in either case, and at least one card will be played, then they will certainly lose this turn.
	\end{solutions}

Let us model this argument symbolically.  Say $L$ is the statement ``I play the first card'', $R$ is the statement ``I play the second card'' and $C$ is the statement ``My friend loses the game''.  We know $L \vee R$ is true and we are trying to argue $C$.  We did so by arguing that $L \implies C$ \textbf{and} $R \implies C$!

This is the idea behind the argument form called \index{disjunction elimination}\textbf{disjunction elimination} or \textbf{proof by cases}:

If we know that $L \vee R$ is true, and we want to argue $C$ is true, then we need to argue $(L \implies C) \wedge (R \implies C)$.   We do that by following the natural deduction proof outline for conjugation and implication.  We often label these two implications as ``cases'', and call this kind of argument a ``case analysis''.

\begin{fitch*}
		\textrm{Given $L \vee R$.}\\
		\textrm{Case 1:  Assume $L$}\\
		\fa \textrm{Argue $C$}\\
		\textrm{Case 2:  Assume $R$}\\
		\fa \textrm{Argue $C$}\\
		\textrm{Conclude that $C$ is true}
	\end{fitch*}


\begin{example}
		Lets prove that if $x<-4$ or $x>5$, then $x^2 > 9$.
		
		Symbolically, we are trying to show
		
		\[
		\forall x \in \mathbb{R}: [(x< -4) \vee (x>5)] \implies (x^2 >9)
		\] 
		
		\begin{fitch}
				\textrm{Let $x_1 \in \mathbb{R}$ be arbitrary.}\\
				\textrm{Assume $(x< -4) \vee (x>5)$.}\\
				\fa \textrm{Case 1:  Assume $x< -4$.}\\
				\fa \fa \textrm{Then $x^2 > 16$, so $x^2 > 9$.}\\
				\fa \textrm{Case 2:  Assume $x>5$.}\\
				\fa \fa \textrm{Then $x^2> 25$, so $x^2 > 9$.}
			\end{fitch}
		
		Commentary:
		
		\begin{itemize}
				\item In line 1 we are using Universal introduction.
				\item In line 2 we are using Implication introduction.  We are assuming the hypothesis of the implication.
				\item In the remaining lines we are performing a case analysis.  In both cases, we conclude that $x^2>9$.
			\end{itemize}
	\end{example}


\subsection{Proving a disjunction}

To prove a disjunction it is sufficient to prove just one of the disjuncts.  This is called \textbf{disjunction introduction}.  Since there are two disjuncts, there are two proof outlines (one for the left disjunct, and one for the right disjunct).


Disjunction Introduction (left)

\begin{fitch*}
	\textrm{Argue $L$}\\
	\textrm{Conclude $L \vee R$}
\end{fitch*}

Disjunction Introduction (right)

\begin{fitch*}
	\textrm{Argue  $L$}\\
	\textrm{Conclude $L \vee R$}
\end{fitch*}

Many theorems in mathematics have disjunctions in both the hypothesis and the conclusion.  In these cases, it is frequently true that in some cases we would prove one disjunct of the conclusion, and in other cases we would prove the other.

\begin{example}
	Lets start with a really basic example and prove that the following sentence is true:
	
	\[
	(\textrm{$6$ is odd}) \vee (\textrm{$6$ is even})
	\]
	
	\begin{proof}
$6$ is even because $6 = 2 \cdot 3$. 
	\end{proof}
	
	This is a complete proof!  To prove a disjunction, you need only prove one of the disjuncts.
	\end{example}

\begin{example}
		Now for a more complicated example.  Lets prove that if $10$ or $15$ divides $n$, then $2$ or $3$ divides $n$. Symbolically 
		
		\[
		\forall n \in \mathbb{Z}: [(10 \divides n) \vee (15 \divides n)] \implies [(2 \divides n) \vee (3 \divides n)]
		\]
		
		\begin{fitch}
				\textrm{Let $n_1 \in \Z$ be arbitrary}\\
				\textrm{Assume  $[(10 \divides n_1) \vee (15 \divides n_1)]$}\\
				\fa \textrm{Case 1:  Assume $10 \divides n_1$}\\
				\fa \fa \textrm{Then $n_1 = 10k$ for some integer $k$.  Choose one such and name it $k_1$}\\
				\fa \fa \textrm{So $n_1 = 10k_1$}\\
				\fa \fa \textrm{So $n_1 = 2(5k_1)$}\\
				\fa \fa \textrm{So $2 \divides n_1$}\\
				\fa \fa \textrm{So $[(2 \divides n) \vee (3 \divides n)]$}\\
				\fa \textrm{Case 2:  Assume $15 \divides n_1$}\\
				\fa \fa \textrm{Then $n_1 = 15k$ for some integer $k$.  Choose one such and name it $k_2$}\\
				\fa \fa \textrm{So $n_1 = 15k_2$}\\
				\fa \fa \textrm{So $n_1 = 3(5k_2)$}\\
				\fa \fa \textrm{So $3 \divides n_1$}\\
				\fa \fa \textrm{So $[(2 \divides n) \vee (3 \divides n)]$}\\
			\end{fitch}
	\end{example}

	Commentary:
	
	\begin{itemize}
			\item In line 1 we are using universal introduction.  We need to argue using a generic element.
			\item In line 2 we are using implication introduction.  We need to assume our hypothesis and try to argue our conclusion.
			\item In lines 3 and 9 we are breaking our argument into cases.  This is disjunction elimination.  We need to reach the same conclusion that  $[(2 \divides n) \vee (3 \divides n)]$ in both cases.
			\item We are able to move from line $7$ to line $8$, and from line $13$ to line $14$, by using disjunction introduction.  Since we proved one disjunct, we proved the disjunction.
		\end{itemize}


\section{Negation}

\begin{xca}
	Which of these sentences are true and which are false?
	
	\begin{enumerate}
			\item ``The moon is made of cheese''
			\item ``The sentence `the moon is made of cheese' is true. ''
			\item ``The sentence `the moon is made of cheese' is false. ''
			\item ``The moon is in orbit around the Earth''
			\item ``The sentence `The moon is in orbit around the Earth' is true. ''
			\item ``The sentence `The moon is in orbit around the Earth' is false. ''
		\end{enumerate}
	\end{xca}

\begin{solutions}
	
	\begin{enumerate}
	\item ``The moon is made of cheese'' - False.
	\item ``The sentence `the moon is made of cheese' is true. '' - False.  The sentence ``The moon is made of cheese'' is false, not true.
	\item ``The sentence `the moon is made of cheese' is false. '' - True!  Then sentence ``The moon is made of cheese'' is false as claimed.
	\item ``The moon is in orbit around the Earth''  - True.
	\item ``The sentence `The moon is in orbit around the Earth' is true. '' -True.  Then sentence ``The moon is in orbit around the Earth'' is true.
	\item ``The sentence `The moon is in orbit around the Earth' is false. '' - False.  The sentence ``The moon is in orbit around the Earth'' is true, not false.
\end{enumerate}
\end{solutions}

The idea of negation is that the negation of a statement $P$ is a new statement $\neg P$ which says ``$P$ is false''.

However, this is inconvenient as a definition of negation.  Instead, we will use the following definition:

\begin{definition} Let $P$ be a sentence.  We define the \index{Negation}\textbf{negation} of $P$ by the following symbolic formula
	
	\[
	\neg P = (P \implies \F)
	\]
	
\end{definition}

This definition agrees with the idea we had above.  If $P$ is true, then the statement ``$P$ is false'' is false, which agrees with  $(T \implies F) = F$.  If $P$ is false, then the statement ``$P$ is false'' is true, which agrees with $(F \implies F) = T$.

Our definition of negation might seem more familiar when you think about how you argue a negation.  It is normal, in our everyday experience, to show that a claim is false by arguing that it leads to absurd conclusions.  For instance if someone claims that they ate $100$ pounds of food yesterday, I would argue the negation of that statement by saying ``If you did eat $100$ pounds of food, your stomach would explode and you would die.  However, you are alive before me.  Thus you must not have eaten $100$ pounds of food.''

When we want to argue a negation of a statement, we naturally assume the statement and then make a valid argument to reach a conclusion which we know to be false.  This is the intuition which is captured by our definition.

Since negation is defined in terms of implication, we use and prove it according to those same rules for implication.  Let's spell that out a bit, and see some examples.

\subsection{Using negations}

We already know that modus ponens is elimination rule for implication.   What does modus ponens look like when applied to $\neg P = P \implies \F$?  It says that if we know $P$ and $\neg P$, then we can derive $F$.  

%Consider the following abstract situation.  Say we know $\neg P$, $P \vee Q$, and $Q \implies C$ are all true.  We want to argue that $C$ is unconditionally true.  We will argue via case analysis on $P \vee Q$.
%
%\begin{fitch}
%		\textrm{Given $\neg P$, $P \vee Q$, $Q \implies C$.}
%		\textrm{Case 1:  Assume $P$}\\
%		\fa \textrm{ Then $P$ and $\neg P = (P \implies F)$}\\
%		\fa \textrm{Then $F$ by Modus Ponens.}\\
%		\fa \textrm{$F \implies C$ is vacuously true.}\\
%		\fa \textrm{We can then conclude $C$ by Modus Ponens.}\\
%		\textrm{Case 2:  Assume $Q$}\\
%		\fa \textrm{Then $Q$ and $Q \implies C$}\\
%		\fa \textrm{So $C$ is true by Modus Ponens}\\
%		\textrm{Since $C$ is true in either case, we can conclude $C$.}
%	\end{fitch} 
%
%In less formal language we might say ``Case 1 never happens so we do not have to worry about it''.  Here we instead use the principle of  to say that since Case 1 leads to an absurdity, it also leads to $C$.  Then since we can ``legitimately'' derive $C$ in Case 2, we are able to conclude $C$ absolutely by disjunctive elimination.

Here is an example of how this sort of reasoning would arise in practice.  We will need to make use of the intuitively clear statement that among any three consecutive numbers, one of them is divisible by $3$.  While this makes sense, proving this statement relies on mathematical induction, which we will explore in a later chapter.


\begin{example}
		Let's try to prove that if an integer $n$ is not divisible by $3$, then $n^2 -1$ is divisible by $3$.
		
		Symbolically
		
		\[
		\forall n \in \mathbb{Z}: \neg(3\divides n) \implies 3 \divides (n^2 - 1)
		\]
		
		\begin{fitch}
				\textrm{Let $n_1 \in \mathbb{Z}$ be chosen arbitrarily.}\\
				\textrm{Assume $\neg(3 \divides n)$}\\
				\fa \textrm{Either $n-1$, $n$, or $n+1$ is divisible by $3$. [Known Theorem.]}\\
				\fa \textrm{Assume $n$ is divisible by $3$}\\
				\fa \fa \textrm{Then $3 \divides n$ and $\neg (3 \divides n) = (3 \divides n \implies \F)$.}\\
				\fa \fa \textrm{By Modus Ponens, we can conclude $\F$.}\\
				\fa \fa \textrm{$\F \implies (3 \divides (n^2 - 1))$ is vacuously true.}\\
				\fa \fa \textrm{By Modus Ponens, $3 \divides (n^2 - 1)$ is true.}\\
				\fa \textrm{Assume $n-1$ is divisible by $3$}\\
				\fa \fa \textrm{Then there is an integer $k$ for which $n-1 = 3k$.}\\
				\fa \fa \textrm{So $(n-1)(n+1) = 3k(n+1)$.}\\
				\fa \fa \textrm{So $n^2- 1 = 3(kn+k)$}\\
				\fa \fa \textrm{So $3 \divides (n^2 - 1)$, since $kn+k \in \mathbb{Z}$}\\
				\fa \textrm{Assume $n+1$ is divisible by $3$}\\
				\fa \fa \textrm{Then there is an integer $k$ for which $n+1 = 3k$.}\\
				\fa \fa \textrm{So $(n-1)(n+1) = 3k(n-1)$.}\\
				\fa \fa \textrm{So $n^2- 1 = 3(kn-k)$}\\
				\fa \fa \textrm{So $3 \divides (n^2 - 1)$, since $kn-k \in \mathbb{Z}$}
			\end{fitch}
		
		Commentary:
		
		\begin{enumerate}
				\item Universal introduction
				\item Implication introduction
				\item This is the unproven fact we mentioned we would need.  It is a disjunction.
				\item On this line, and line (9) and (14), we are initiating a proof of $3 \divides (n_1^2 - 1)$ via disjunctive elimination.
				\item We have both the assumption at line (2) and (4) active here.
				\item The assumptions are contradictory.
				\item This line and the next are reexplaining the ``principle of explosion''.
				\item Since we have $\F$ and $\F \implies (3 \divides (n_1^2 - 1))$,  Modus Ponens finishes this argument in this case.
				\item The rest of the proof should be fairly accessible to you at this point.
			\end{enumerate}
	\end{example}
     
\subsection{Proving negations}

Our definition of negation is $\neg P = (P \implies \F)$.  So our introduction rule for negation will follow the rule for proving an implication:

\begin{fitch*}
	\textrm{Assume $P$}\\
	\fa \textrm{Argue $\F$.}\\
	\textrm{Conclude $\neg P$}
\end{fitch*}

Often when we ``argue $\F$'', we will be reaching an \textrm{absurdity}, i.e. a statement which must be false for purely logical reasons.  So we might assume $P$, make some arguments relative to that assumption, and eventually conclude something like ``$n$ is an integer and $n$ is not an integer'', or some other statement which must be false.

\begin{example}
	
	Lets show that $6$ is not odd.  Symbolically
	
	\[
	\neg( \textrm{$6$ is odd})
	\]
	
	\begin{fitch}
		\textrm{Assume $6$ is odd.}\\
		 \fa \textrm{Then $6 = 2k+1$ for some integer $k$.  Pick one such and name it $k_1$.}\\
		\fa \textrm{Then $6 = 2k_1 + 1$.}\\
		\fa \textrm{Then $k_1 = 2.5$.}\\
		\fa \textrm{So $2.5$ is an integer, which is clearly false.}
		\end{fitch}
	
	Commentary:
	
	\begin{enumerate}
			\item To prove the negation of a statement, we must assume the statement and derive a contradiction.  This is negation introduction.
			\item Under the assumption that $6$ is odd, this is valid. We use existential quantifier elimination when we pick the witness $k_1$.
			\item ...
			\item Algebra.
			\item Starting from the premise that $6$ is odd, we have reached a clearly false conclusion.  So $6$ cannot be odd.
		\end{enumerate}
	
	It is important to realize that this proof structure (assuming a proposition and deriving an absurdity) is the \textbf{only} way to establish a negation.
	
	A common incorrect solution to this problem would be to write:
	
	\begin{quote}[WARNING: INCORRECT WORK]
		
\begin{fitch*}
	6 = 2(2.5)+1\\
	\textrm{But an odd number is defined to be a number of the form  $2k+1$ where $k \in \Z$.}\\
	\textrm{$2.5$ is not  an integer.}\\
	\textrm{Thus $6$ is not odd.}
\end{fitch*}

	\end{quote}

This reasoning is faulty.  Let's look at some completely analogous reasoning which leads to an incorrect conclusion.  I will use the same kind of reasoning to show that $2$ is not a rational number!

\begin{fitch*}
		\textrm{$2 = \frac{2\pi}{\pi}$.}\\
		\textrm{But a rational number is defined to be a number of the form  $\frac{a}{b}$ where $a,b \in \Z$.}\\
		\textrm{Neither $2\pi$ nor $\pi$ are integers.}\\
		\textrm{Thus $2$ is not rational.}
	\end{fitch*}

Moral of the story:  to prove the negation of a proposition we \textbf{must} assume the proposition and argue an absurdity.

\subsection{Non-examples}

Whenever we meet a definition of a mathematical concept, it is worthwhile to spend some creative effort inventing your own examples and non-examples.  An example is something which satisfies the definition, and a non-example is something which satisfies the negation of the definition.

For instance, if you take a course on number theory you might encounter the following definition:

\begin{definition}
		A \textbf{perfect number} is an integer greater $1$ which is equal to the sum of its positive divisors, excluding the number itself.
	\end{definition}

When we meet this definition, it is our duty to hunt for examples and non-examples.  In this case we can just conduct a systematic seach:

\begin{itemize}
		\item $2$ is not a perfect number.  If it were, then the sum of its divisors would be $2$.  But $2$ only has $1$ as a divisor, and $1 \neq 2$.
		\item $3$, $4$, $5$ are also not perfect numbers by similar arguments.
		\item $6$ is a perfect number because the positive divisors of $6$ are $1$, $2$, $3$ and $6$.  So the sum of all of these positive divisors (excluding $6$ itself) is $1+2+3 = 6$.
	\end{itemize}

As the non-example $2$ shows, when we argue that something is a non-example, we need to follow our strategy for proving negations.
	
	
	
	
	
	
	\end{example}








%% $Id: fitch.sty,v 1.6 2003/06/28 16:53:00 johanw Exp $

% Macros for Fitch-style formal proofs
% Johan W. Kl�?wer, June 10, 2001

% EDITS (Alexander W. Kocurek, June 8, 2019)
% %\RequirePackage{mdwtab,latexsym,amsmath,amsfonts,ifthen}
% - too many fonts were loading
% - removed mdwtab, which redefines tabular, causing several conflicts such as overriding color in tabular cells and redefining \hline so that \hline without arguments no longer works

\RequirePackage{mathtools,amsmath,ifthen,array}

% Line height in proofs
\newlength{\fitchlineht}
\setlength{\fitchlineht}{1.5\baselineskip}
% Horizontal indent between proof levels
\newlength{\fitchindent}
\setlength{\fitchindent}{0.7em}
% Indent to comment
\newlength{\fitchcomind}
\setlength{\fitchcomind}{2em}
% Line number width
\newlength{\fitchnumwd}
\setlength{\fitchnumwd}{1em}

% Altered from mdwtab.sty: shorter vline, for start of subproof
\makeatletter
\newcommand\fvline[1][\arrayrulewidth]{\vrule\@height.5\fitchlineht\@width#1\relax}
\makeatother
% Ordinary vertical line
\newcommand{\fa}{\vline\hspace*{\fitchindent}}
% Vertical line, shorter: Use at start of (sub)proof
\newcommand{\fb}{\fvline\hspace*{\fitchindent}}
% Hypothesis
\newcommand{\fh}{\fvline%
  \makebox[0pt][l]{{%
      \raisebox{-1.4ex}[0pt][0pt]{\rule{1.5em}{\arrayrulewidth}}}}%
  \hspace*{\fitchindent}}
% Hypothesis, with longer vert line: for >1 hypothesis
\newcommand{\fj}{\vline%
  \makebox[0pt][l]{{%
      \raisebox{-1.4ex}[0pt][0pt]{\rule{1.5em}{\arrayrulewidth}}}}%
  \hspace*{\fitchindent}}
% Modal subproof: takes argument = operator
\newcommand{\fitchmodal}[1]{% 
  \makebox[0pt][r]{${}^{#1}$\,}\fvline\hspace*{\fitchindent}}
\newcommand{\fn}{\fitchmodal{\Box}}% Box subproof 
\newcommand{\fp}{\fitchmodal{\Diamond}}% Diamond subproof
% Modal subproof with hypothesis in first line (as in Fitch)
\newcommand{\fitchmodalh}[1]{% 
  \makebox[0pt][r]{${}^{\footnotesize #1}$\,}%
  \fvline%
  \makebox[0pt][l]{{%
      \raisebox{-1.4ex}[0pt][0pt]{\rule{1.5em}{\arrayrulewidth}}}}%
  \hspace*{\fitchindent}}
\newcommand{\fm}{\fitchmodalh{\Box}}% Box subproof with hypothesis
\newcommand{\fq}{\fitchmodalh{\Diamond}}% Diamond subproof with hypothesis
% Rule: formula introduction marker. \fr with line, \fs without line
\newcommand{\fr}{%
  \makebox[0pt][r]{${\rhd}$\,\,}\vline\hspace*{\fitchindent}}
\newcommand{\fs}{%
  \makebox[0pt][r]{${\rhd}$\,\,}}
% Box around argument, like new variable in ql
\newcommand{\fw}[1]{\fbox{\footnotesize $#1$}}

% 
\newcounter{fitchcounter}
\setcounter{fitchcounter}{0}
%To avoid starting from 1, \setboolean{resetfitchcounter}{false}
\newboolean{resetfitchcounter}
\setboolean{resetfitchcounter}{true}
%To avoid increasing numbers, \setboolean{increasefitchcounter}{false}
\newboolean{increasefitchcounter}
\setboolean{increasefitchcounter}{true}
%\formatfitchcounter can be altered if need be, though only once per proof
\newcommand{\formatfitchcounter}[1]{\scriptsize \arabic{#1}}
%Typeset the counter
\newcommand{\fitchcounter}{%
  \ifthenelse{\boolean{increasefitchcounter}}{\addtocounter{fitchcounter}{1}}{}
  \formatfitchcounter{fitchcounter}}
%A line with a special number -- a tag, e.g. \ftag{\vdots}{}
\newcommand{\ftag}[2]{\multicolumn{1}%
  {!{\makebox[\fitchnumwd][r]{\footnotesize $#1$}\hspace{\fitchindent}}>{$}l<{$}@{\hspace{\fitchcomind}}}%
  {#2}} % Arc put the tag in math mode by default

% Main environments
% Arc modified the next two environments by replacing Ml (which is supposed to force math mode, left aligned in mdwtab) with >{$}l<{$}, which has the same effect in tabular
\newenvironment{fitchnum}%
{\ifthenelse{\boolean{resetfitchcounter}}{\setcounter{fitchcounter}{0}}{}
  \begin{tabular}{!{\makebox[\fitchnumwd][r]{\fitchcounter }\hspace{\fitchindent}}>{$}l<{$}@{\hspace{\fitchcomind}}l}}%
{\end{tabular}}

\newenvironment{fitchunum}%
{\begin{tabular}{!{\makebox[\fitchnumwd][r]{}\hspace{\fitchindent}}>{$}l<{$}@{\hspace{\fitchcomind}}l}}%
{\end{tabular}}

\newenvironment{fitch}{\renewcommand{\arraystretch}{1.5}
  \begin{fitchnum}}{\end{fitchnum}}
\newenvironment{fitch*}{\renewcommand{\arraystretch}{1.5}
  \begin{fitchunum}}{\end{fitchunum}}

% The following is useful for giving a numbered formula, then the proof.
\newenvironment{flem}[2]%
{\begin{eqnarray}
    &#1\label{#2}\\
    &\begin{fitch}}%
    {\end{fitch}\notag\end{eqnarray}}

%To write comment field for two consecutive lines, with brace
\newcommand{\ftwocom}[1]{%
  \parbox[t]{3cm}{
    \raisebox{-.6\baselineskip}[\baselineskip][0pt]{%
      $\left.
        \begin{aligned}
          \,\\ \,
        \end{aligned}
      \right\}$\quad #1}
  }}
  
%\chapter{Induction and Recursion}

\section{Introduction}

Mathematics is, in large part, the study of patterns.  Consider the following facts:

\begin{enumerate}
		\item $6^1 - 1 = 5$
		\item $6^2 - 1 = 35$
		\item $6^3 -1 = 215$
		\item $6^4 - 1 = 1295$
	\end{enumerate}

Looking at these four examples seems to suggest a pattern.  It seems that $6^n - 1$ is always a multiple of $5$.  You can continue experimenting to confirm that the pattern continues to hold.  However, no matter how many computations you do, you cannot prove that $6^n-1$ is \textit{always} divisible by $5$ this way.  Even if you test up to $6^{10000} - 1$, there is still a possibility that $6^{10000} - 1$ might not be divisible by $5$.

The ordered list of number $5, 35, 215, 1295, \dots$ is an example of a \textbf{sequence}.  We have a formula for this sequence $f(n) = 6^n -1$, for $n \in \{1,2,3,4, \dots\}$.  In other words, a sequence is just a function whose domain is a subset of the set of natural numbers.

In this section we will learn how to work with sequences which are presented explicitly by formulas and recursively by recurrence relations.  We will learn how to use sigma notation to work with sums of sequences.

We will learn a new method of proof called ``Mathematical Induction'' which will allow us to prove theorems about the natural numbers, including the theorem that $6^n - 1$ is always divisible by $5$.

We will apply mathematical induction to prove a theorem called the Quotient Remainder theorem.  This theorem is important in number theory, and is an essential prerequisite to proving even basic theorems like the fact that every integer is either even or odd.

Finally we will give some proof techniques which are equivalent to mathematical induction called ``strong induction'' and ``the well ordering principle''.  While these are equivalent to ordinary mathematical induction, the equivalence is not obvious, and these techniques are produce proofs with a different flavor from ordinary induction.

\section{Sequences}

\begin{definition}
	Let $I$ be a set of consecutive integers.  For instance, we could have a finite set like $I = \{2,3,4,5\}$ or an infinite set like $I = \{5, 6, 7, 8, \dots\}$.  We call a function from $I$ to any set $X$ a \index{Sequence}\textbf{sequence} of elements of $X$ indexed by $I$.  The set $I$ is called the \index{Index Set}\textbf{index set} of the sequence, and the individual elements of $I$ are called \index{Index}\textbf{indices} (singular:  \textbf{index}).
\end{definition}

Note:  While it is common to use function notation for sequences (such as writing $f: \mathbb{N} \to \mathbb{R}$ defined by $f(n) = n^2$), it is also conventional to use a letter with a subscript for the index variable.  For instance, it is common to write $a_n = n^2, n \in \mathbb{N}$.  It is normal to substitute particular values into the index, as in $a_7  =49$.

One way to present a sequence is to give the index set and a formula (also called a \index{Closed Form Expression}\textbf{closed form expression}) for the value of the sequence at a generic input.

\begin{xca}
		Write the first four terms of each of the following sequences:
		
		\begin{enumerate}
				\item $a_n = n^2 -1, n \in \{4,5,6,\dots\}$
				\item $b_n = \cos(\pi n), n \in \mathbb{N}$
				\item $c_n = (-1)^n, n \in \mathbb{N}$
				\item $a_n = \frac{n(n+1)}{2}, n \in \mathbb{N}$
			\end{enumerate}
	\end{xca}

\begin{xca}
	Below are the first 4 terms of a sequence.  Try and find a closed form expression which fits.  Note that the index set is given, so that the first element of the index set should map to the first term in the sequence.
	
	\begin{enumerate}
		\item $2, 4, 6, 8$, where  $I = \mathbb{N}$.
		\item $2, 4, 6, 8$,  where $I = \{5, 6, 7, 8, \cdots\}$
		\item $3, 8, 15, 24$, where $I = \{1,2,3, \cdots\}$
	\end{enumerate}
\end{xca}

We can combine functions through arithmetic operations and composition, and sequence are functions, so we can also do these things with sequences:

\begin{xca}
		Let $I = \mathbb{N}$
		\begin{enumerate}
			\end{enumerate}
	\end{xca}

\section{Sums}

\section{Induction}

\section{The Quotient-Remainder Theorem}

\subsection{Why every integer is either even or odd}

\section{Strong Induction and the Well Ordering Principle}

%\chapter{Sets, Relations, and Functions}

\section{Relations}
\subsection{Equivalence Relations}
\subsection{Modular Arithmetic}
\subsection{Partial Orders}

\section{Functions}
\subsection{Function composition}
\subsection{Injective functions}
\subsection{Surjective functions}


\section{Sets}
\subsection{Set Relations and Operations}
\subsubsection{Subsets}
\subsubsection{Intersection}
\subsubsection{Union}
\subsubsection{Compliment}
\subsubsection{Cartesian Products}
\subsubsection{Power Sets}
\subsection{Cardinality}
%\chapter{Graphs and Trees}
%\chapter{Graphs and Trees}

\section{Graphs}
\section{Trees}
%\chapter{Basics of Number Theory}

%\appendix
%    Include appendix "chapters" here.
%\include{}

\backmatter
%    Bibliographies can be prepared with BibTeX using amsplain,
%    amsalpha, or (for author-year style) natbib.
\bibliographystyle{amsalpha}
\bibliography{}

%    See note above about multiple indexes.
\printindex

\end{document}

%%%%%%%%%%%%%%%%%%%%%%%%%%%%%%%%%%%%%%%%%%%%%%%%%%%%%%%%%%%%%%%%%%%%%%%%

%    Templates for common elements in a book; for information specific
%    to this series, see the sample amstext-doc.pdf and .tex files.
%    For additional general information, see the AMS Author Handbook,
%    Monograph Classes, included in the author package, and the amsthm
%    user's guide, linked from http://www.ams.org/tex/amslatex.html .

%    Section headings
\section{}
\subsection{}

%    Exercises grouped in a section
\begin{xcb}
\begin{enumerate}
\item ...
\item ...
\end{enumerate}
\end{xcb}

%    Exercise standing alone in text
\begin{exa}[Optional exercise heading]
% text of exercise
\end{exa}

%    Framed environment for highlighting important (brief) information
\begin{framedthm}{<type of theorem-class element>}
% text
\end{framedthm}

%    Environment for inclusions to be skipped on first reading; note
%    that, even if there is no heading text, the second pair of braces
%    must be present.
\begin{inclusion}{<optional heading text>}
% text, will be set in smaller type
\end{inclusion}

%    Figure insertion; default placement is top; if the figure occupies
%    more than 75% of a page, the [p] option should be specified.
\begin{figure}
\includegraphics{filename}
\caption{text of caption}
\label{}
\end{figure}

%    Ordinary theorem and proof
\begin{theorem}[Optional addition to theorem head]
% text of theorem
\end{theorem}

\begin{proof}[Optional replacement proof heading]
% text of proof
\end{proof}

%    Marker for the end of some important element.  \lozenge is used
%    here only as an illustration; other symbols may also be used.
% text
\xqed{\lozenge}

%    Mathematical displays; for additional information, see the amsmath
%    user's guide:  texdoc amamath  or
%  http://mirror.ctan.org/tex-archive/macros/latex/required/amsmath/amsldoc.pdf

%    Numbered equation
\begin{equation}
\end{equation}

%    Unnumbered equation
\begin{equation*}
\end{equation*}

%    Aligned equations
\begin{align}
  &  \\
  &
\end{align}

%    In-chapter bibliography
\renewcommand{\bibname}{References for this chapter}
\begin{inchapterbibliography}{AAA}
\bibitem[]{} text
\end{inchapterbibliography}

%-----------------------------------------------------------------------
% End of amstext-l-template.tex
%-----------------------------------------------------------------------
