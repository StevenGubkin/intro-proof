\chapter{Tautologies, Equivalence, and Additional Proof Techniques}

\section{Enumerating Truth Assignments}

In this chapter we will be concerned with all the possible truth values which a proposition like $P \vee (Q \implies R)$ could take depending on the truth value of each \textbf{atomic proposition}  $P$, $Q$, and $R$ in the expression.  Figuring out all of the distinct available choices is difficult.  We could just start writing them down ``willy nilly''

		\begin{table}[h!]
	\begin{center}
		\caption{``Willy Nilly'' truth assignments}
		\begin{tabular}{c|c|c} 
			$P$ & $Q$ & $R$ \\
			\hline
			$\F$ & $\F$ & $\T$ \\ 
			$\F$ & $\T$ & $\F$ \\ 
			$\T$ & $\F$ & $\F$ \\ 
			$\F$ & $\T$ & $\T$ \\ 
			$\vdots$ & $\vdots$ & $\vdots$ \\ 
		\end{tabular}
	\end{center}
\end{table}

This strategy has several shortcomings:

\begin{enumerate}
		\item It is hard to know that you have listed all possibilities.
		\item It is hard to check that you have not accidentally listed a combination more than once.
		\item Even if you get all of them exactly once, the particular ordering you come up with will not be easily reproducible in the future.
		\end{enumerate}
	
These shortcomings indicate that we might be better off \textbf{organizing} the way we are listing these truth assignments.

Since each atomic proposition can be either true or false, we have $2$ choices for $P$.  Each of those $2$ choices for $P$ leads to $2$ choices for $Q$.  This is $4$ total so far.  Each of these $4$ assignments leads to $2$ choices for $R$, for a total of $8$ possibilities.  Let's try to list them all out in an organized, reproducible manner.  We will start by making a table with 8 rows:

		\begin{table}[h!]
	\begin{center}
		\caption{Starting to organize the truth assignments}
		\begin{tabular}{c|c|c} 
			$P$ & $Q$ & $R$ \\
			\hline
			&  &  \\  \hline
			&  &  \\  \hline
			&  &  \\  \hline
			&  &  \\  \hline
            &  &  \\  \hline
            &  &  \\  \hline
            &  &  \\  \hline
            &  &  \\  \hline
		\end{tabular}
	\end{center}
\end{table}

We know that half of the assignments will have $P$ false, and the rest will have $P$ true.  We will list the false assignments first.


		\begin{table}[h!]
	\begin{center}
		\caption{Sorting by $P$ first}
		\begin{tabular}{c|c|c} 
			$P$ & $Q$ & $R$ \\
			\hline
			\F &  &  \\  \hline
			\F &  &  \\  \hline
			\F &  &  \\  \hline
			\F &  &  \\  \hline
			\T &  &  \\  \hline
			\T &  &  \\  \hline
			\T &  &  \\  \hline
			\T &  &  \\  \hline
		\end{tabular}
	\end{center}
\end{table}

If $P$ is false, then there are $4$ choices.  Of these, half should have $Q = \F$ and the rest should have $Q = \T$.  Similar remarks apply if $P$ is true.


		\begin{table}[h!]
	\begin{center}
		\caption{Sorting by $Q$ second}
		\begin{tabular}{c|c|c} 
			$P$ & $Q$ & $R$ \\
			\hline
			\F & \F &  \\  \hline
			\F & \F &  \\  \hline
			\F & \T &  \\  \hline
			\F & \T &  \\  \hline
			\T & \F &  \\  \hline
			\T & \F &  \\  \hline
			\T & \T &  \\  \hline
			\T & \T  &  \\  \hline
		\end{tabular}
	\end{center}
\end{table}

Finally, look at the rows in groups of two.  In each group of two, the truth value of $P$ and $Q$ are identical, and we have exactly two choices for the truth value of $R$.  Let's fill those in:


		\begin{table}[h!]
	\begin{center}
		\caption{Sorting by $R$ last}
		\begin{tabular}{c|c|c} 
			$P$ & $Q$ & $R$ \\
			\hline
			\F & \F & \F  \\  \hline
			\F & \F & \T \\  \hline
			\F & \T & \F \\  \hline
			\F & \T & \T \\  \hline
			\T & \F & \F \\  \hline
			\T & \F &  \T\\  \hline
			\T & \T &  \F\\  \hline
			\T & \T  &  \T \\  \hline
		\end{tabular}
	\end{center}
\end{table}

This way of building this collection of possible truth assignments has mitigated all of our concerns about the ``willy nilly'' approach:

\begin{enumerate}
	\item It is easy to know that you have listed all possibilities.
	\item It is easy to check that you have not accidentally listed a combination more than once.
	\item This ordering is easily reproducible in the future.
\end{enumerate}


\begin{xca}

Make a table showing all possible truth value assignments for $4$ atomic propositions $A$, $B$, $C$, $D$.  Use the same organizing principles we agreed on when creating the three column table.
	
\end{xca}

\begin{solutions}
		We can think of each choice of truth value for $A$, $B$, $C$ and $D$ as creating a fork in the road, doubling our number of possible ``paths''.
		
		

		\begin{center}
			\scalebox{0.5}{
			\begin{forest}
				 [[${A = \F}$ [${B = \F}$ [${C = \F}$[${D = \F}$][${D = \T}$]][${C = \T}$[${D = \F}$][${D = \T}$]]][${B = \T}$ [${C = \F}$[${D = \F}$][${D = \T}$]][${C = \T}$[${D = \F}$][${D = \T}$]]]][ ${A = \T}$[${B = \F}$ [${C = \F}$[${D = \F}$][${D = \T}$]][${C = \T}$[${D = \F}$][${D = \T}$]]][${B = \T}$ [${C = \F}$[${D = \F}$][${D = \T}$]][${C = \T}$[${D = \F}$][${D = \T}$]]]]]
			\end{forest}
		}
		\end{center}
	
	Then reading from left to right we obtain
	
			\begin{table}[h!]
		\begin{center}
			\begin{tabular}{c|c|c|c} 
				$A$ & $B$ & $C$  & $D$ \\
				\hline
				$\F$ & $\F$ & $\F$  & $\F$ \\ \hline
				$\F$ & $\F$ & $\F$  & $\T$ \\ \hline
				$\F$ & $\F$ & $\T$  & $\F$ \\ \hline
				$\F$ & $\F$ & $\T$  & $\T$ \\ \hline
				$\F$ & $\T$ & $\F$  & $\F$ \\ \hline
				$\F$ & $\T$ & $\F$  & $\T$ \\ \hline
				$\F$ & $\T$ & $\T$  & $\F$ \\ \hline
				$\F$ & $\T$ & $\T$  & $\T$ \\ \hline
				$\T$ & $\F$ & $\F$  & $\F$ \\ \hline
				$\T$ & $\F$ & $\F$  & $\T$ \\ \hline
				$\T$ & $\F$ & $\T$  & $\F$ \\ \hline
				$\T$ & $\F$ & $\T$  & $\T$ \\ \hline
				$\T$ & $\T$ & $\F$  & $\F$ \\ \hline
				$\T$ & $\T$ & $\F$  & $\T$ \\ \hline
				$\T$ & $\T$ & $\T$  & $\F$ \\ \hline
				$\T$ & $\T$ & $\T$  & $\T$ \\ 
			\end{tabular}
		\end{center}
	\end{table}
	
	\end{solutions}

These ``truth assignment tables'' bear some similarity to lists of numbers.  If fact, if we replace $\F$ with $0$ and $\T$ with $1$ we obtain the following table when we use $3$ atomic propositions:

		\begin{table}[h!]
	\begin{center}
		\caption{3 bit binary numbers}
		\begin{tabular}{c|c|c} 
			$P$ & $Q$ & $R$ \\
			\hline
			0& 0& 0  \\  \hline
			0& 0 & 1 \\  \hline
			0 & 1 & 0 \\  \hline
			0 & 1 & 1 \\  \hline
			1 & 0 & 0 \\  \hline
			1 & 0 &  1 \\  \hline
			1 & 1 &  0 \\  \hline
			1 & 1 &  1 \\  \hline
		\end{tabular}
	\end{center}
\end{table}

We really, to make these tables, we are just counting in binary!

For the purposes of reading the rest of this book, you do not need to know anything about binary or other bases.  You \textbf{do} need to be able to produce a truth assignment table for a set of atomic propositions in the way we have discussed in this section.

Understanding other bases is a rite of passage in mathematics and computer science culture.  Understanding the binary (base $2$)  and hexidecimal (base $16$) systems are important for computer scientists because (at its heart) a computer is ``really'' doing all of its computations using binary (``off"/``on'') circuitry.  So number systems based on powers of $2$ are important to computer scientists.

For these reasons, although it is a bit of a digression from the main thrust of this text, we will take some time in this section to explore counting and computing in other bases.

\subsubsection{A digression on other bases}


	

\section{Tautologies and Equivalence}

\begin{xca}
Consider the following four statements:  $P \implies Q$,  $(\neg P) \vee Q$, $P \vee (\neg Q)$, and $(\neg Q) \implies (\neg P)$. 

A few entries of this table have been filled our correctly as a model.  Complete the rest of the table.

		\begin{table}[h!]
	\begin{center}
		\begin{tabular}{c|c|c|c|c|c|c|c} 
			$P$ & $Q$ & $\neg P$ & $\neg Q$  & $P \implies Q$ & $(\neg P) \vee Q$ &  $P \vee (\neg Q)$ & $(\neg Q) \implies (\neg P)$ \\
			\hline
			 $\F$& $\F$ & $\T$ &         &          &            &  $\T$   &   \\ \hline
			 $\F$& $\T$ &         & $\F$ &          &            &            &  $\T$  \\ \hline
			 $\T$& $\F$ &         &         &  $\F$ &            &            &     \\ \hline
			 $\T$& $\T$ &         &         &          &  $\T$   &            &      \\ 
		\end{tabular}
	\end{center}
\end{table}

\end{xca}

\begin{solutions}
	
	\begin{table}[h!]
		\begin{center}
			\begin{tabular}{c|c|c|c|c|c|c|c} 
				$P$ & $Q$ & $\neg P$ & $\neg Q$  & $P \implies Q$ & $(\neg P) \vee Q$ &  $P \vee (\neg Q)$ & $(\neg Q) \implies (\neg P)$ \\
				\hline
				$\F$& $\F$ & $\T$ & $\T$ & $\T$ & $\T$ & $\T$  &  $\T$  \\ \hline
				$\F$& $\T$ & $\T$ & $\F$ & $\T$ & $\T$ & $\F$  &  $\T$  \\ \hline
				$\T$& $\F$ & $\F$ & $\T$ & $\F$ & $\F$ & $\T$  &  $\F$   \\ \hline
				$\T$& $\T$ & $\F$ & $\F$ & $\T$ & $\T$ & $\T$ &  $\T$    \\ 
			\end{tabular}
		\end{center}
	\end{table}
\end{solutions}

Notice that the columns for $P \implies Q$,  $(\neg P) \vee Q$, and $(\neg Q) \implies (\neg P)$ are all identical.  That means that no matter what the truth value of $P$ or $Q$, the truth value of these three expressions will always agree! 

\begin{definition}
		Two expressions involving $n$ letters (standing for atomic propositions) and the logical connectives $\wedge$, $\implies$, $\bi$, $\neg$, and $\vee$ are called \index{Truth Functionally Equivalent}\textbf{truth functionally equivalent} or \index{Logically Equivalent}\textbf{logically equivalent} if the two expressions evaluate to the same truth value for all of the $2^n$ assignments of the atomic propositions.
		
		In other words, they are equivalent if when we make a truth table involving all $2^n$ rows the columns corresponding to the two expressions are identical.
		
		If $S_1$ and $S_2$ are two statements, we will write $S_1 \equiv S_2$ to indicate that they are truth functionally equivalent.
	\end{definition}

\begin{xca}
		For each pair of expressions, determine whether they are truth functionally equivalent or not.
		
		TODO
	\end{xca}

\begin{solutions}
	For each pair of expressions, determine whether they are truth functionally equivalent or not.
	
	TODO
\end{solutions}

\begin{definition}
		An expression $S$ involving $n$ letters (standing for atomic propositions) and the logical connectives $\wedge$, $\implies$, $\bi$, $\neg$, and $\vee$ is called a \index{Tautology}\textbf{tautology} if it is truth functionally equivalent to $T$.  
		
		It is called a \index{Contradiction}\textbf{contradiction} if it is truth functionally equivalent to $F$.
		
		It is called  \index{Conditionally True}\textbf{conditionally true} if it is neither a tautology nor a contradiction.
		
		In other words, $S$ is a tautology if its column in a truth table has only $\T$, it is a contradiction if this column consists only of $\F$, and it is conditionally true if there is a mix of $\T$ and $\F$ entries.
	
	\end{definition}

Note:  We could have defined tautology first, and then defined $S_1 \equiv S_2$ if $S_1 \bi S_2$ is a tautology.  Think about why!

\begin{xca}
		Which of the following statements are tautologies, which are contradictions, and which are conditionally true?
		\begin{enumerate}
			\item $p \implies p$
	\end{enumerate}

TODO
	\end{xca}

\begin{xca}
	Which of the following statements are tautologies, which are contradictions, and which are conditionally true?
	\begin{enumerate}
		\item $p \implies p$
	\end{enumerate}
	
	TODO
\end{xca}

\subsection{Establishing Tautologies using Natural Deduction}

Consider the tautology $(p \wedge r) \implies ( q \implies r )$.

\begin{xca}
		Create a truth table to confirm that $(p \wedge r) \implies ( q \implies r )$ is a tautology.
	\end{xca}

\begin{solutions}
	TODO
	\end{solutions}

We can understand why this is a tautology intuitively:  if we know that $p \wedge r$ is true, then we know that $r$ is true.  So no matter what $q$ is, $q \implies r$ will be true.  

We can formalize this argument using natural deduction.  Starting from no hypotheses at all, we can deduce $(p \wedge r) \implies ( q \implies r )$ is a true statement as follows:

\begin{fitch}
	\textrm{Assume $p \wedge r$}\\
	\fa \textrm{$p$ and $r$ are both true by assumption.}\\
	\fa \textrm{Assume $q$}\\
	\fa \fa \textrm{$r$ is still true by assumption.}
	\end{fitch}

Commentary:

\begin{enumerate}
	\item Implication introduction.
	\item Conjunction elimination.
	\item Implication introduction.
	\item We know $r$ is true here because we are still assuming $r$ from line $2$.
	\end{enumerate}

This shows that no matter what the statements $p$, $q$, and $r$ are, we can argue that $(p \wedge r) \implies ( q \implies r )$ is true.  This establishes that $(p \wedge r) \implies ( q \implies r )$ is a tautology without needing to construct a truth table.  This kind of reasoning is closer to how mathematicians identify and use tautologies in their work:  I think it is rarely the case that a mathematician will create a truth table to help them assess the validity of an argument.  We mostly confirm whether the reasoning presented conforms to natural deduction rules (although, this is often done subconsciously).

The following is a theorem of mathematical logic which is beyond the scope of this book, but provides a bridge between truth tables (boolean calculus) and natural deduction:

\begin{theorem}
	If one can create an argument for a proposition in natural deduction without any hypotheses, then that proposition is a tautology.
\end{theorem}

This should make sense:  it says that if we can argue that a proposition must be true using natural deduction, then the proposition will in fact be true for any values of the propositional variables.  We do not provide a proof here because it would involve making fine distinctions between syntax (form) and semantics (meaning) which are too sophisticated for this text.  It would also involve reasoning using mathematical induction, which we will not introduce until later.

\begin{xca}
		Show that each of the following is a tautology using natural deduction:
		
		TODO
	\end{xca}

Warning:  It is not true that every tautology can be obtained via the natural deduction rules we have covered so far.  $P \vee (\neg P)$ is a tautology called the law of the excluded middle. It cannot be proven using our rules.  To introduce a disjunction, we have to be able to argue one of the disjuncts.  Absent knowledge about $P$, we do not know which disjunct to argue, and so we cannot prove $P \vee (\neg P)$  using our natural deduction rules.  The next section will explore this in further depth.

 
\subsection{The Law of the Excluded Middle }

We have made a lot of progress together!  This is worth celebrating.

We have learned how to make some fine distinctions between logical concepts which you may have glossed over in the past.  We have learned to calculate with Boolean connectives, which is a sort of ``truth calculus''.  We have learned how to structure a logical argument based on the logical form of the statement we are trying to argue.  We have been telling these as two parallel, but connected stories (``Boolean computation'' vs. ``Natural Deduction'').  We will see that there is a bit of controversy surrounding where these two paths diverge.

One major divide is between \index{classical}\textbf{classical} mathematicians and \index{intuitionist}\textbf{intuitionist} mathematicians.

I am not an expert on philosophy, but I can summarize the difference as I understand it.  A classical mathematician believes that if two expressions are truth functionally equivalent, then we should feel free to use them interchangeably in arguments.  For instance, the classical mathematician believes that to argue ``If it is raining, then I get wet" it is enough to argue that ``If I do not get wet, then it is not raining''.  One statement is of the form $P \implies Q$, while the other is of the form $\neg Q \implies \neg P$.  Since these are truth functionally equivalent, the classical mathematician would accept an argument for one in place of the other.  In particular, a classical mathematician is comfortable using any (truth functional) tautology as part of their arguments.

An intuitionist mathematician only trusts the natural deduction rules we have already established.  They do allow some substitutions, but would only accept an argument for $S_2$ in place of an argument for $S_1$ if you could argue $S_2$ from $S_1$ using the rules.   The intuitionist does not accept (Boolean) truth functional equivalents as valid substitutions in arguments.  For instance, an intuitionist would accept an argument for $P \implies (Q \wedge R)$ in place of an argument that $(P \implies Q) \wedge (P \implies R)$ since if you know  $P \implies (Q \wedge R)$ then you can argue $(P \implies Q) \wedge (P \implies R)$ as follows:

\begin{fitch}
		\textrm{Assume $P$}\\
		\fa 	\textrm{Since $P \implies (Q \wedge R)$, we know $Q \wedge R$ - Implication elimination}\\
		\fa \textrm{Since we know $Q \wedge R$, we know $Q$ - Conjunction Elimination}\\
		\textrm{Lines $1$ through $3$ give us $P \implies R$ by Implication introduction.}\\
		\textrm{Assume $P$}\\
		\fa 	\textrm{Since $P \implies (Q \wedge R)$, we know $Q \wedge R$ - Implication elimination}\\
		\fa \textrm{Since we know $Q \wedge R$, we know $R$ - Conjunction Elimination}\\
		\textrm{Lines $5$ through $7$ give us $P \implies R$ by Implication introduction}\\
		\textrm{Lines $4$ and $8$ give us $(P \implies Q) \wedge (P \implies R)$ by Conjunction introduction}.
	\end{fitch}

The difference between the two camps boils down to the following tautology, accepted by the classical mathematician as part of any argument, but rejected by the intuitionist as part of their arguments:

\begin{theorem}[The Law of the Excluded Middle]
	$P \vee (\neg P)$ is a tautology. 
	\end{theorem}

The classical mathematician is comfortable with introducing this tautology anywhere in their argument.  They would have no issue claiming ``$7$ is odd or $7$ is not odd'' as part of an argument.  An intuitionist only accepts a disjunction like this if they can actually argue one of the two disjuncts.  Intuitionistic arguments are more demanding for this reason.

You might say ``What is the big deal intuitionists?  It seems obvious to me that either $7$ is odd or $7$ is not odd!  What problem do you have with this? ''.  More extreme uses of the Excluded Middle might give you pause.  For instance, there are some conjectures which have been open for hundreds of years (such as the famous ``Riemann Hypothesis''$ =$ RH ).  Imagine you want to argue a conjecture $C$  is true.  You give one argument if the RH is true.  Then you give a completely different argument if RH is false.  Since it is true in both cases, you argue (disjunctive elimination) that the theorem is true irrespective of the validity of the RH.  This feels unsatisfying.  We have shown that two paths lead us to the same place, but we have not justified that either path is open to us.  The intuitionist makes these distinctions very clearly.  They would accept the theorems $\textrm{RH} \implies C$ and $(\neg \textrm{RH}) \implies C$, but would not yet accept $C$ as a theorem until RH is decided.

Most mathematicians adopt a classical perspective in their work.  Whatever your philosophical commitments end up being, I find that it is worthwhile to pay attention to the intuitionistic validity of your arguments.  As a matter of taste, I find that arguments which can be made intuitionistically tend to be ``cleaner'', and often provide more insight.  However, for the rest of this text we will adopt a classical perspective.  We may note when a given argument is intuitionistically valid (or not) from time to time.

The following theorem (whose proof is beyond the level of this text) provides support that the law of the excluded middle is the point of contention between these two camps:

\begin{theorem}
		Every Boolean tautology can be established by natural deduction if one assumes the law of the excluded middle as a hypothesis.
\end{theorem}


The next three sections in this chapter explore argument forms which are valid classically, but not intuitionistically.  They had to wait for a discussion of truth functional equivalence to be introduced.  The final section in this chapter is intuitionistically valid.  We only delayed introducing it because it didn't fit naturally into the flow of the previous chapter.

\section{Proof by Contraposition}

As noted earlier, the statements $P \implies Q$ and $(\neg Q) \implies (\neg P)$ are truth functionally equivalent. 

\begin{xca}
		Confirm again that $ [ P \implies Q] \equiv (\neg Q) \implies (\neg P)$ using a truth table.
	\end{xca}

\begin{solutions}
	TODO
\end{solutions}

We can also confirm that $ [ P \implies Q] \bi [(\neg Q) \implies (\neg P)]$ is a tautology using natural deduction, if we also permit the use of excluded middle:

\begin{fitch}
	\fj	\textrm{Assume $P \implies Q$} & \\ %1
	\fa \fh \textrm{Assume $\neg Q$} & \\ %2
	\fa \fa \fh \textrm{Assume $P$} & \\ %3
	\fa \fa \fa \fa  \textrm{$Q$} & By (1), (3), and modus ponens.\\ %4
	\fa \fa \fa \fa \F & By (2), (4)\\ %5
	\fa \fa \fa \neg P & $\neg$ intro, (3) and (4)--(5).\\ %6
	\fa \fa  \neg Q \implies \neg P & $\implies$ intro (2) and (3) -- (6)\\ %7
	 \fa [ P \implies Q] \implies [(\neg Q) \implies (\neg P)] & $\implies$ intro (1) and (2) -- (7)\\ %8
	\fj \textrm{Assume $(\neg Q) \implies (\neg P)$ } & \\ %9
	\fa \fh \textrm{Assume $P$} & \\ %10
	\fa \fa \fa Q \vee \neg Q & Excluded middle \\ %11
	\fa \fa \fa \textrm{Assume $Q$}  & \\ %12
	\fa \fa \fa \fa Q & by (12) \\ %13
	\fa \fa \fa \textrm{Assume $\neg Q$} \\  %14
	\fa \fa \fa \fa \neg P & $\implies$ elim, (9) and (14)\\ %15
	\fa \fa \fa \fa \F & (15) and (10) \\ %16
	\fa \fa \fa \fa Q & (16) and principle of explosion \\ %17
	\fa \fa \fa  Q& $\vee$ elim (11), (12)--(13), and (14)--(17) \\ %18
	\fa \fa P \implies Q & $\implies$ intro (10), (11)--(18)\\ %19
	\fa [(\neg Q) \implies (\neg P)] \implies (P \implies Q) & $\implies$ intro (9), (10)--(19)\\ %20
	 \ftag{~}{\fs [ P \implies Q] \bi [(\neg Q) \implies (\neg P)]} & $\bi$ introduction (1) and (20) %21
	\end{fitch}

Notice that $[ P \implies Q] \implies [(\neg Q) \implies (\neg P)]$ is intuitionistically valid, while  $[(\neg Q) \implies (\neg P)] \implies (P \implies Q)$ depends on excluded middle.

\begin{definition}
		The \index{Contrapositive}\textbf{contrapositive} of the implication  $P \implies Q$ is the implication $(\neg Q) \implies (\neg P)$ .  
	\end{definition}

Note:  An implication is truth functionally equivalent to its contrapositive.


The statement $P \implies Q$ is \textbf{not} equivalent to either $Q \implies P$ or  $(\neg P) \implies (\neg Q)$.  However, $Q \implies P$ is equivalent to   $(\neg P) \implies (\neg Q)$,  since the second is the contrapositive of the first.



\begin{definition}
		The \index{Converse}\textbf{converse} of the implication $P \implies Q$ is the implication $Q \implies P$.  The \index{Inverse}\textbf{inverse} of the implication $P \implies Q$ is the implication $(\neg P) \implies (\neg Q)$.  
	\end{definition}

Note:  An implication is not logically equivalent to either its converse or inverse.  However, the converse and inverse of an implication are logically equivalent to each other.  Also, while the term ``inverse'' is standard in the literature, my personal experience is that this term is not used often.  The terms ``contrapositive'' and ``converse'' are both in widespread use.

\begin{xca}
	Confirm that $P \implies Q$ is not logically equivalent to $Q \implies P$, but that $Q \implies P$ is equivalent to $(\neg P) \implies (\neg Q)$.
\end{xca}

\begin{solutions}
TODO
\end{solutions}


\begin{xca}
		For each of the following pairs of implications, decide if the second is the contrapositive, converse, or inverse of the second.
		
		TODO	
\end{xca}


\section{Proof by Contradiction}

\begin{theorem}
		$(\neg P) \implies \F \equiv P$
	\end{theorem}

If we argue that $P$ is true by arguing the logically equivalent $(\neg P ) \implies \F$, we are using \index{Proof by Contradiction}\textbf{proof by contradiction}.
\begin{xca}
		Verify that $(\neg P) \implies \F$ is logically equivalent to  $P$ using a truth table.
	\end{xca}

\begin{solutions}
				\begin{table}[h!]
			\begin{center}
				\begin{tabular}{c|c|c} 
					$P$ & $\neg P$ & $(\neg P) \implies \F$ \\
					\hline
					$\F$ & $\T$ & $\F$ \\ 
					$\T$ & $\F$ & $\T$ \\ 
				\end{tabular}
			\end{center}
		\end{table}
	\end{solutions}

If we accept the law of the excluded middle, we can also derive this equivalence through natural deduction:

\begin{fitch}
	 	\textrm{Assume $P$}\\
	 	\fa  \textrm{Assume $\neg P$}\\
	 	\fa \fa \textrm{$P \wedge (\neg P) = \F$}\\
	 	\textrm{Assume $(\neg P) \implies F$}\\
	 	\fa \textrm{$P \vee (\neg P)$}\\
	 	\fa \textrm{Assume $P$}\\
	 	\fa \fa \textrm{Then $P$}\\
	 	\fa \textrm{Assume $\neg P$}\\
	 	\fa \fa \textrm{Then $\F$}\\
	 	\fa \fa \textrm{Then $P$}
	\end{fitch}

\begin{example}
		TODO
	\end{example}

\section{Proving Disjunctions using Implications}

\begin{theorem}
		$P \vee Q \equiv (\neg P) \implies Q$
\end{theorem}

\begin{xca}
		Show that this theorem is true by building an appropriate truth table.
	\end{xca}

\begin{solutions}
	
			\begin{table}[h!]
		\begin{center}
			\begin{tabular}{c|c|c|c|c} 
				$P$ & $Q$ & $\neg P$ & $(\neg P) \implies Q$  & $P \vee Q$ \\
				\hline
				$\F$ & $\F$ & $\T$ & $\F$  & $\F$ \\
				$\F$ & $\T$ & $\T$ & $\T$  & $\T$ \\
				$\T$ & $\F$ & $\F$ & $\T$  & $\T$ \\
				$\T$ & $\T$ & $\F$ & $\T$  & $\T$ \\
		\hline
			\end{tabular}
		\end{center}
	\end{table}
	\end{solutions}

So, at least in classical mathematics, we may prove a disjunction by instead proving that the negation of one disjunct implies the other.

This is a fairly normal way to reason in everyday life as well.  Most people would consider the following two sentences to have the same meaning:

\begin{itemize}
		\item You will either do your homework or you will suffer the consequences.
		\item If you do not do your homework, then you will suffer the consequences.
	\end{itemize}

\begin{xca}
		TODO
	\end{xca}

\section{Proving Uniqueness}

Every other proof technique in this section depended on acceptance of the law of the excluded middle, and so was classically valid but not intuitionistically valid.  This final section is arguably not even really a proof technique:  it is just a special piece of mathematical language which comes up often enough to pay it some special attention.

 If $A(x)$ is a predicate, we want to express ``There is a unique $x \in \mathcal{U}$ such that $A(x)$''.

We will translate this as follows:

\[
(\exists x \in \mathcal{U}: A(x)) \wedge (\forall x_1, x_2 \in \mathcal{U}: A(x_1) \wedge A(x_2) \implies x_1 = x_2  )
\]

This warrants some explanation.

Requiring that $(\exists x \in \mathcal{U}: A(x)) $ is straightforward enough:  we do want \textbf{at least} one $x \in \mathcal{U}$ which makes $A(x)$ true.

The next part is trickier.  Let's unpack the meaning of $(\forall x_1, x_2 \in \mathcal{U}: A(x_1) \wedge A(x_2) \implies x_1 = x_2  )$.

This is saying that \textbf{if} $A(x_1)$ and $A(x_2)$ are both true, then $x_1$ must be $x_2$.  This is a tricky way of saying ``only one works'' while using only the logical concepts (universal quantification, implication, equality) which we have already developed.

\begin{xca}
		TODO
	\end{xca}