\chapter{Induction and Recursion}

\section{Introduction}

Mathematics is, in large part, the study of patterns.  Consider the following facts:

\begin{enumerate}
		\item $6^1 - 1 = 5$
		\item $6^2 - 1 = 35$
		\item $6^3 -1 = 215$
		\item $6^4 - 1 = 1295$
	\end{enumerate}

Looking at these four examples seems to suggest a pattern.  It seems that $6^n - 1$ is always a multiple of $5$.  You can continue experimenting to confirm that the pattern continues to hold.  However, no matter how many computations you do, you cannot prove that $6^n-1$ is \textit{always} divisible by $5$ this way.  Even if you test up to $6^{10000} - 1$, there is still a possibility that $6^{10000} - 1$ might not be divisible by $5$.

The ordered list of number $5, 35, 215, 1295, \dots$ is an example of a \textbf{sequence}.  We have a formula for this sequence $f(n) = 6^n -1$, for $n \in \{1,2,3,4, \dots\}$.  In other words, a sequence is just a function whose domain is a subset of the set of natural numbers.

In this section we will learn how to work with sequences which are presented explicitly by formulas and recursively by recurrence relations.  We will learn how to use sigma notation to work with sums of sequences.

We will learn a new method of proof called ``Mathematical Induction'' which will allow us to prove theorems about the natural numbers, including the theorem that $6^n - 1$ is always divisible by $5$.

We will apply mathematical induction to prove a theorem called the Quotient Remainder theorem.  This theorem is important in number theory, and is an essential prerequisite to proving even basic theorems like the fact that every integer is either even or odd.

Finally we will give some proof techniques which are equivalent to mathematical induction called ``strong induction'' and ``the well ordering principle''.  While these are equivalent to ordinary mathematical induction, the equivalence is not obvious, and these techniques are produce proofs with a different flavor from ordinary induction.

\section{Sequences}

\begin{definition}
	Let $I$ be a set of consecutive integers.  For instance, we could have a finite set like $I = \{2,3,4,5\}$ or an infinite set like $I = \{5, 6, 7, 8, \dots\}$.  We call a function from $I$ to any set $X$ a \index{Sequence}\textbf{sequence} of elements of $X$ indexed by $I$.  The set $I$ is called the \index{Index Set}\textbf{index set} of the sequence, and the individual elements of $I$ are called \index{Index}\textbf{indices} (singular:  \textbf{index}).
\end{definition}

Note:  While it is common to use function notation for sequences (such as writing $f: \mathbb{N} \to \mathbb{R}$ defined by $f(n) = n^2$), it is also conventional to use a letter with a subscript for the index variable.  For instance, it is common to write $a_n = n^2, n \in \mathbb{N}$.  It is normal to substitute particular values into the index, as in $a_7  =49$.

One way to present a sequence is to give the index set and a formula (also called a \index{Closed Form Expression}\textbf{closed form expression}) for the value of the sequence at a generic input.

\begin{xca}
		Write the first four terms of each of the following sequences:
		
		\begin{enumerate}
				\item $a_n = n^2 -1, n \in \{4,5,6,\dots\}$
				\item $b_n = \cos(\pi n), n \in \mathbb{N}$
				\item $c_n = (-1)^n, n \in \mathbb{N}$
				\item $a_n = \frac{n(n+1)}{2}, n \in \mathbb{N}$
			\end{enumerate}
	\end{xca}

\begin{xca}
	Below are the first 4 terms of a sequence.  Try and find a closed form expression which fits.  Note that the index set is given, so that the first element of the index set should map to the first term in the sequence.
	
	\begin{enumerate}
		\item $2, 4, 6, 8$, where  $I = \mathbb{N}$.
		\item $2, 4, 6, 8$,  where $I = \{5, 6, 7, 8, \cdots\}$
		\item $3, 8, 15, 24$, where $I = \{1,2,3, \cdots\}$
	\end{enumerate}
\end{xca}

We can combine functions through arithmetic operations and composition, and sequence are functions, so we can also do these things with sequences:

\begin{xca}
		Let $I = \mathbb{N}$
		\begin{enumerate}
			\end{enumerate}
	\end{xca}

\section{Sums}

\section{Induction}

\section{The Quotient-Remainder Theorem}

\subsection{Why every integer is either even or odd}

\section{Strong Induction and the Well Ordering Principle}
